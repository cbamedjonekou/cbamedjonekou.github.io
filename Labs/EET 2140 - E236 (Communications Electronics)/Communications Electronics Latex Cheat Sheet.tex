\documentclass[10pt,landscape]{article}
\usepackage{multicol}
\usepackage{calc}
\usepackage{ifthen}
\usepackage[landscape]{geometry}
\usepackage{amsmath,amsthm,amsfonts,amssymb}
\usepackage{color,graphicx,overpic}
\usepackage{hyperref}


\pdfinfo{
  /Title (example.pdf)
  /Creator (TeX)
  /Producer (pdfTeX 1.40.0)
  /Author (Seamus)
  /Subject (Example)
  /Keywords (pdflatex, latex,pdftex,tex)}

% This sets page margins to .5 inch if using letter paper, and to 1cm
% if using A4 paper. (This probably isn't strictly necessary.)
% If using another size paper, use default 1cm margins.
\ifthenelse{\lengthtest { \paperwidth = 11in}}
    { \geometry{top=.2in,left=.2in,right=.2in,bottom=.2in} }
    {\ifthenelse{ \lengthtest{ \paperwidth = 297mm}}
        {\geometry{top=1cm,left=1cm,right=1cm,bottom=1cm} }
        {\geometry{top=1cm,left=1cm,right=1cm,bottom=1cm} }
    }

% Turn off header and footer
\pagestyle{empty}

% Redefine section commands to use less space
\makeatletter
\renewcommand{\section}{\@startsection{section}{1}{0mm}%
                                {-1ex plus -.5ex minus -.2ex}%
                                {0.5ex plus .2ex}%x
                                {\normalfont\large\bfseries}}
\renewcommand{\subsection}{\@startsection{subsection}{2}{0mm}%
                                {-1explus -.5ex minus -.2ex}%
                                {0.5ex plus .2ex}%
                                {\normalfont\normalsize\bfseries}}
\renewcommand{\subsubsection}{\@startsection{subsubsection}{3}{0mm}%
                                {-1ex plus -.5ex minus -.2ex}%
                                {1ex plus .2ex}%
                                {\normalfont\small\bfseries}}
\makeatother

% Define BibTeX command
\def\BibTeX{{\rm B\kern-.05em{\sc i\kern-.025em b}\kern-.08em
    T\kern-.1667em\lower.7ex\hbox{E}\kern-.125emX}}

% Don't print section numbers
\setcounter{secnumdepth}{0}


\setlength{\parindent}{0pt}
\setlength{\parskip}{0pt plus 0.5ex}

%My Environments
\newtheorem{example}[section]{Example}
% -----------------------------------------------------------------------

\begin{document}
\raggedright
\footnotesize
\begin{multicols}{3}


% multicol parameters
% These lengths are set only within the two main columns
%\setlength{\columnseprule}{0.25pt}
\setlength{\premulticols}{1pt}
\setlength{\postmulticols}{1pt}
\setlength{\multicolsep}{1pt}
\setlength{\columnsep}{2pt}

\begin{center}
     \Large{\underline{Com. Elec. Formula Sheet}} \\
\end{center}

\section{Decibel Formulas}
\subsection{Relative Power Gain}

$$A_P = \frac{P_O}{P_I}$$
\text{where } $P_O$ \text{ and } $P_I$ \text{ are defined as the following:}
$$P_I = \frac{V_I^2}{R_I}$$
$$P_O = \frac{V_O^2}{R_O}$$

\subsection{Relative Voltage Gain}

$$A_V = \frac{V_O}{V_I}$$

\subsection{Relative Power Gain in dB}

$$A_P(db) = 10 \log_{10}A_P$$
$$\text{Given that } R_O = R_I$$  
\text{ If } $R_O \neq R_I$ \text{ then the general form is given by the following:}
$$A_P(db) = 10 \log_{10}\big(\frac{\frac{V_O^2}{R_O}}{\frac{V_I^2}{R_I}}\big)$$

\subsection{Relative Voltage Gain in dB}

$$A_V(db) = 20 \log_{10}\big(\frac{V_O}{V_I}\big) = 20 \log_{10} A_V$$
$$\text{ If } R_O \neq R_I \text{ then the general form is given by the following:}$$
$$A_V(db) = 20 \log_{10}\big(\frac{V_O}{V_I}\big) - 10 \log_{10}\big(\frac{R_O}{R_I}\big)$$  

\subsubsection{Special Case}
$$\text{If }R_O \neq R_I\text{ then the general form is given by the following:}$$  
$$A_V(db) = 10 \log_{10}\big(\frac{V_O^2}{V_I^2}\big) - 10 \log_{10}\big(\frac{R_O}{R_I}\big)$$
$$ = 20 \log_{10}\big(\frac{V_O}{V_I}\big) - 10 \log_{10}\big(\frac{R_O}{R_I}\big)$$

\subsection{Absolute Power Gain dBm}

$$A_{P(dBm)} = 10 \log_{10} \big(\frac{P}{1 \text{ mW}}\big), \text{ dBm}$$

\subsection{Absolute Power Gain dBw}

$$A_{P(dBw)} = 10 \log_{10} \big(\frac{P}{1 \text{ W}}\big), \text{ dBw}$$

\subsection{Signal-to-Noise Ratio}

$$\text{SNR } = 10 \log_{10}\big(\frac{\text{Signal Power}}{\text{Noise Power}}\big)$$ 
$$\text{And given that } R_O = R_I,$$
$$\text{SNR } = 10 \log_{10} \big(\frac{V_S^2}{V_N^2}\big)$$
$$\text{SNR } = 20 \log_{10} \big(\frac{V_S}{V_N}\big)\text{dB}$$

\subsection{Impulse Noise}

$$ dB_S = 20 \log_{10} \big(\frac{P}{0.0002 \space \bar{\mu}}\big) \text{, where P is sound pressure in } \bar{\mu}$$
$$ \bar{\mu} = 1 \space \frac{\text{dyne}}{\text{cm}^2} = 10^{-6} \text{ of atmospheric pressure at sea level}$$

\subsection{Gaussian (White) Noise}

$$P_n = kT\Delta f $$
$$k = \text{ Boltzmann's Constant }(1.38 * 10^{-23})\text{ J/K}$$
$$T = \text{ resistor temperature in Kelvin }(K)$$
$$\Delta f = \text{ system bandwidth.}$$

\subsubsection{Gaussian (White) Noise Formulas}

$\text{Using the above proportionality we can relate}$
$\text{bandwidth to noise, shown below:}$
$\text{Given the noise can be represented as }e_n$
$\text{then we can say the following:}$

$$P_n = \frac{V_n^2}{R} =  kT\Delta f, \text{ where } V_n = \frac{e_n}{2}$$

$\text{This is true by Ohm's law. By solving in terms of }e_n$
$\text{we get the following:}$

$$\frac{V_n^2}{R} =  kT\Delta f \text{ where } V_n = \frac{e_n}{2}$$
$$\frac{\big(\frac{e_n}{2}\big)^2}{R} = \frac{\big(\frac{e_n^2}{4}\big)}{R} = kT\Delta f$$
$$\big(\frac{e_n^2}{4}\big) =  kT\Delta f R$$
$$e_n =  \sqrt{4kT\Delta f R}$$


\subsection{Noise Ratio}

$$NF = 10 \log_{10} \frac{\frac{S_i}{N_i}}{\frac{S_o}{N_o}} = 10 \log_{10} NR$$  
$$NR = \frac{\frac{S_i}{N_i}}{\frac{S_o}{N_o}} \text{ is the Noise Ratio}$$  
$$\frac{S_i}{N_i} = \text{ input SNR}$$  
$$\frac{S_o}{N_o} = \text{ output SNR}$$    

\subsection{Reactance Noise Effects}

$$\Delta f_{eq} = \frac{\pi}{2} BW$$
$$BW = 3 \text{ dB; bandwidth for } RC, \space LC, \text{ or }RLC \text{ circuits.}$$  

\subsection{Noise Created by Amplifiers in Cascade}

$$NR = NR_1 + \frac{NR_2 - 1}{P_{G_1}} + ... + \frac{NR_n - 1}{P_{G_1} * P_{G_2} * P_{G_{(n-1)}}}$$  
$$NR = \text{overall noise ratio of } n \text{ stages}.$$  
$$P_G = \text{power gain ratio}$$  

\subsection{Equivalent Noise Temperature}

$$T_{eq} = T_0(NR-1)$$

$\text{where } T_0 = 290 \text{ K, a reference temperature in Kelvin.} $

\subsection{Equivalent Noise Resistance}
$\text{Sometimes used by Manufacturers to represent the}$
$\text{noise generated by a device with a fictitious resistance.}$  
$\text{The following represents this: }$

$$R_{eq} = \sqrt{4kT\Delta f R}$$  

\section{Modulation Index}

$$m = \frac{E_i}{E_c}$$  
$$\%m = \frac{E_i}{E_c} * 100\%$$  
$$\%m = \frac{B-A}{B+A} * 100\%$$  
$$B = \text{ AM Waveform}$$  
$$A = \text{ The Minimum Peak-to-Peak value}$$  

\subsection{Overmodulation}

$$\%m = \frac{B-O}{B+O} * 100\%$$  
$$B = \text{ AM Waveform}$$  
$$O = \text{ The Minimum Peak-to-Peak value} \leq 0$$  

\subsection{Amplitude Modulation/Mixing in Frequency Domain}

\subsubsection{Carrier Signal}

$$e_c = E_C \sin \omega_{c}t$$  
$$\text{where }e_i = \text{ is the instantaneous value of the carrier}$$  
$$E_C = \text{ is the maximum peak value of the carrier when unmodulated}$$  
$$\omega = 2 \pi f \text{ ("f" is the carrier frequency)}$$ 
$$t = \text{ is a unit of measure}$$ 

\subsubsection{Information Signal}

$$e_i = E_I \sin \omega_{i}t$$  
$$\text{where }e_i = \text{ is the instantaneous value of the information}$$  
$$E_i = \text{ is the maximum peak value of the intelligence}$$
$$\text{when unmodulated}$$  
$$\omega = 2 \pi f \text{ ("f" is the carrier frequency)}$$ 
$$t = \text{ is a unit of measure}$$ 

\subsubsection{AM Modulated Waveform}

$$e = E_c \sin \omega_{c}t + \frac{mE_c}{2} \cos (\omega_{c} - \omega_{i})t - \frac{mE_c}{2} \cos (\omega_{c} + \omega_{i})t$$  
$1. \space E_c \sin \omega_{c}t \textbf{ relates to the carrier } (1)$

$2. \space \frac{mE_c}{2} \cos (\omega_{c} - \omega_{i})t \textbf{ relates to the the}$
$\textbf{lower sideband at } f_c - f_i \space(2)$  

$3. \space \frac{mE_c}{2} \cos (\omega_{c} + \omega_{i})t \textbf{ relates to the the}$
$\textbf{upper sideband at } f_c + f_i \space(3)$

\subsection{Power Distribution in Carriers and Sidebands}

$$E_{SF} = \frac{mE_C}{2}$$  
$$E_{SF} = \text{ side frequency amplitude}$$  
$$m = \text{ modulation index}$$  
$$E_C = \text{ carrier amplitude}$$  

\subsubsection{Total Transmitted Power}

$$P_{t} = P_c \left( 1 +\frac{m^2}{2} \right)$$  
$$P_{t} = \text{ Total Transmitted Power (sidebands and carrier)}$$  
$$m = \text{ modulation index}$$  
$$P_c = \text{ carrier power}$$  

\subsubsection{Total Transmitted Current}

$$I_{t} = \text{ Total Transmitted Current (sidebands and carrier)}$$  
$$I = \text{ modulation index}$$  
$$I_c = \text{ carrier current}$$  

\section{Frequency Modulation}

\subsection{How FM Generator Works? "The Concept of Deviation"}

$$f_{OUT} = f_C + ke_i$$  
$$f_{OUT} = \text{ instantaneous output frequency}$$  
$$f_C = \text{ output carrier frequency}$$  
$$k = \text{ deviation constant [kHz/V]}$$  
$$e_i = \text{ modulating (intelligence) input}$$  

\subsection{Quick Facts}

$\text{1. Deviation constant defines how much carrier frequency}$ 
$\text{will deviate for input voltage level.}$
$\text{2. Deviation constant dependent on system design.}$
$\text{3. Knowing deviation on either side of carrier is essential for}$ 
$\text{determining occupied bandwidth of modulated signal.}$

\subsection{Direct FM}

$\text{Direct FM involves messing w/ the frequency component}$
$\text{of a sinusoidal wave:}$
$$f \text{ in } \omega \space (\space A_P \sin (\omega t + \theta) = A_P \sin (2 \pi f t + \theta) \space)$$ 

\subsection{Indirect FM}

$\text{Indirect FM involves messing w/ the phase angle component}$
$\text{of a sinusoidal wave:}$
$$\theta \text{ in the sinusoid}\space (\space A_P \sin (\omega t + \theta) = A_P \sin (2 \pi f t + \theta) \space)$$

\section{FM IN THE FREQUENCY DOMAIN}

$$e = A \sin(\omega_ct + m_f\sin \omega_it)$$  
$$e = \text{ instantaneous voltage}$$  
$$A = \text{ peak value of original carrier wave}$$  
$$\omega_c = \text{ carrier carrier angular velocity } (2\pi f_c)$$     
$$\omega_i = \text{ modulating intelligence signal angular velocity } (2\pi f_i)$$  

\subsection{Modulation Index}

$$m_f = \text{FM Modulation Index} = \frac{\delta}{f_i}$$  
$\delta = \text{maximum frequency shift caused by the}$
$\text{intelligence signal (deviation)}$
$\text{either above or below the carrier; therefore, deviation}$
$\text{written as 3 kHz, for example, has }$
$\delta = \text{ 3 kHz (not 6 kHz) in the above.}$
$f_i = \text{ of the intelligence (modulating) signal}$  

\section{FM Spectrum Analyzer}

$\text{Remember that each Bessel table entry represents}$
$\text{the ratio } \frac{V_2}{V_1} \text{ for its respective carrier}$
$J_0 \text{ or sideband } J_1 \text{ and above signal component, for}$
$\text{a given modulation index.}$
$$P_{dB} = 20 \log_{10} \frac{V_2}{V_1}$$

\section{Power Distribution}

\subsection{Carson's Rule Approximation}

$$BW \cong 2(\delta_{max} + f_{i_{max}})$$

\subsection{Percent of Modulation and Deviation Ratio}

$$DR = \frac{\text{max possible freq deviation}}{\text{max input freq}} = \frac{f_{dev(max)}}{f_{i(max)}}$$

\section{Fourier Series}

$$a_0 = \frac{1}{2 \pi}\int\limits_{-\pi}^{\pi} f(x) dx$$
$$a_n = \frac{1}{ \pi}\int\limits_{-\pi}^{\pi} \cos(nx) dx$$  
$$b_n = \frac{1}{ \pi}\int\limits_{-\pi}^{\pi} \sin(nx) dx$$

\subsection{A0}

$$a_0 = \frac{1}{2 \pi}\int\limits_{-\pi}^{\pi} f(x) dx$$
$$= \frac{1}{2 \pi}\int\limits_{-\pi}^{0} 0 dx + \frac{1}{2 \pi}\int\limits_{0}^{\pi} 0 dx$$
$$= 0 + \frac{1}{2 \pi} = \frac{1}{2}$$

\subsection{An}

$$\text{For } n \geq 1$$
$$a_n = \frac{1}{ \pi}\int\limits_{-\pi}^{\pi} \cos(nx) dx$$
$$= \frac{1}{\pi}\int\limits_{-\pi}^{0} 0 dx + \frac{1}{\pi}\int\limits_{0}^{\pi} \cos(nx) dx$$
$$= 0 + \frac{1}{\pi}\frac{\sin(nx)}{n}\int\limits_{0}^{\pi} = \frac{1}{n\pi}(\sin(n\pi)-\sin(0)) = 
0$$

\subsection{Bn}

$$\text{For } n \geq 1$$
$$b_n = \frac{1}{ \pi}\int\limits_{-\pi}^{\pi} \sin(nx) dx$$
$$= \frac{1}{\pi}\int\limits_{-\pi}^{0} 0 dx + \frac{1}{\pi}\int\limits_{0}^{\pi} \sin(nx) dx$$
$$= -\frac{1}{\pi}\frac{\cos(nx)}{n}\int\limits_{0}^{\pi} = -\frac{1}{n\pi}(\cos(n\pi)-\cos(0)) =$$ 
$$0 \text{ if } n \text{ is even}; \frac{2}{n\pi} \text{ if } n \text{ is odd}$$

% You can even have references
\rule{0.3\linewidth}{0.25pt}
\scriptsize
\bibliographystyle{abstract}
\bibliography{refFile}
\end{multicols}
\end{document}