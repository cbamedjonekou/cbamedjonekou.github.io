\documentclass[10pt,landscape]{article}
\usepackage{multicol}
\usepackage{calc}
\usepackage{ifthen}
\usepackage[landscape]{geometry}
\usepackage{amsmath,amsthm,amsfonts,amssymb}
\usepackage{color,graphicx,overpic}
\usepackage{hyperref}


\pdfinfo{
  /Title (example.pdf)
  /Creator (TeX)
  /Producer (pdfTeX 1.40.0)
  /Author (Seamus)
  /Subject (Example)
  /Keywords (pdflatex, latex,pdftex,tex)}

% This sets page margins to .5 inch if using letter paper, and to 1cm
% if using A4 paper. (This probably isn't strictly necessary.)
% If using another size paper, use default 1cm margins.
\ifthenelse{\lengthtest { \paperwidth = 11in}}
    { \geometry{top=.13in,left=.13in,right=.13in,bottom=.13in} }
    {\ifthenelse{ \lengthtest{ \paperwidth = 297mm}}
        {\geometry{top=1cm,left=1cm,right=1cm,bottom=1cm} }
        {\geometry{top=1cm,left=1cm,right=1cm,bottom=1cm} }
    }

% Turn off header and footer
\pagestyle{empty}

% Redefine section commands to use less space
\makeatletter
\renewcommand{\section}{\@startsection{section}{1}{0mm}%
                                {-1ex plus -.5ex minus -.2ex}%
                                {0.5ex plus .2ex}%x
                                {\normalfont\large\bfseries}}
\renewcommand{\subsection}{\@startsection{subsection}{2}{0mm}%
                                {-1explus -.5ex minus -.2ex}%
                                {0.5ex plus .2ex}%
                                {\normalfont\normalsize\bfseries}}
\renewcommand{\subsubsection}{\@startsection{subsubsection}{3}{0mm}%
                                {-1ex plus -.5ex minus -.2ex}%
                                {1ex plus .2ex}%
                                {\normalfont\small\bfseries}}
\makeatother

% Define BibTeX command
\def\BibTeX{{\rm B\kern-.05em{\sc i\kern-.025em b}\kern-.08em
    T\kern-.1667em\lower.7ex\hbox{E}\kern-.125emX}}

% Don't print section numbers
\setcounter{secnumdepth}{0}


\setlength{\parindent}{0pt}
\setlength{\parskip}{0pt plus 0.5ex}

%My Environments
\newtheorem{example}[section]{Example}
% -----------------------------------------------------------------------

\begin{document}
\raggedright
\footnotesize
\begin{multicols}{3}


% multicol parameters
% These lengths are set only within the two main columns
%\setlength{\columnseprule}{0.25pt}
\setlength{\premulticols}{1pt}
\setlength{\postmulticols}{1pt}
\setlength{\multicolsep}{1pt}
\setlength{\columnsep}{2pt}

\begin{center}
     \Large{\underline{TCET Formula Sheet}} \\
\end{center}

\section{Decibel Formulas}
\subsection{Relative Power Gain}

$$A_P = \frac{P_O}{P_I}$$
\text{where } $P_O$ \text{ and } $P_I$ \text{ are defined as the following:}
$$P_I = \frac{V_I^2}{R_I}$$
$$P_O = \frac{V_O^2}{R_O}$$

\subsection{Relative Voltage Gain}

$$A_V = \frac{V_O}{V_I}$$

\subsection{Relative Power Gain in dB}

$$A_P(db) = 10 \log_{10}A_P$$
$$\text{Given that } R_O = R_I$$  
\text{ If } $R_O \neq R_I$ \text{ then the general form is given by the following:}
$$A_P(db) = 10 \log_{10}\big(\frac{\frac{V_O^2}{R_O}}{\frac{V_I^2}{R_I}}\big)$$

\subsection{Relative Voltage Gain in dB}

$$A_V(db) = 20 \log_{10}\big(\frac{V_O}{V_I}\big) = 20 \log_{10} A_V$$
$$\text{ If } R_O \neq R_I \text{ then the general form is given by the following:}$$
$$A_V(db) = 20 \log_{10}\big(\frac{V_O}{V_I}\big) - 10 \log_{10}\big(\frac{R_O}{R_I}\big)$$

\subsection{Absolute Power Gain dBm}

$$A_{P(dBm)} = 10 \log_{10} \big(\frac{P}{1 \text{ mW}}\big), \text{ dBm}$$

\subsection{Absolute Power Gain dBw}

$$A_{P(dBw)} = 10 \log_{10} \big(\frac{P}{1 \text{ W}}\big), \text{ dBw}$$

\subsection{Signal-to-Noise Ratio}

$$\text{SNR } = 10 \log_{10}\big(\frac{\text{Signal Power}}{\text{Noise Power}}\big)$$ 
$$\text{And given that } R_O = R_I,$$
$$\text{SNR } = 10 \log_{10} \big(\frac{V_S^2}{V_N^2}\big)$$
$$\text{SNR } = 20 \log_{10} \big(\frac{V_S}{V_N}\big)\text{dB}$$

\subsection{Impulse Noise}

$$ dB_S = 20 \log_{10} \big(\frac{P}{0.0002 \space \bar{\mu}}\big) \text{, where P is sound pressure in } \bar{\mu}$$
$$ \bar{\mu} = 1 \space \frac{\text{dyne}}{\text{cm}^2} = 10^{-6} \text{ of atmospheric pressure at sea level}$$

\section{Convert Parallel to Series Resonant Circuits} 
$$R_S = \frac{R_p * X_p^2}{R_p^2 + X_p^2} = \frac{R_p}{Q^2 + 1}$$  
$$X_S = \frac{R_p^2 * X_p}{R_p^2 + X_p^2} = \frac{X_p * Q^2}{Q^2 + 1}$$

\section{Convert Series to Parallel Resonant Circuits} 
$$R_p = \frac{R_S^2 + X_S^2}{R_S}$$  
$$X_p = \frac{R_S^2 + X_S^2}{X_S}$$  

\section{Impedance Matching}

$$R_{SRC} = R_L, \text{ for DC sources.}$$
$$Z_{SRC} = Z_L, \text{ for AC sources.}$$
$$\text{Source Resistance = Load Resistance}$$
$$P_O = P_I * 0.5$$  

\subsection{Key Points L-Pad Networks}  
$\text{1. The primary applications of L-networks involve impedance}$
$\text{matching in RF circuits, transmitters, and receivers.}$  
$\text{2. L-networks are useful in matching one amplifier}$
$\text{output to the input of a following stage.}$  
$\text{3. Any RF circuit application covering a narrow frequency}$  
$\text{range is a candidate for an L-network.}$  
$\text{4. There are four basic versions of the L-network,}$
$\text{with two low-pass versions and two high-pass versions.}$
$\text{5. Most widely used since they attenuate harmonics,}$
$\text{noise, and other undesired signals.}$
$\textbf{6. The impedances that are being matched}$
$\textbf{determine the Q (quality factor) of the circuit,}$
$\textbf{which cannot be specified or controlled.**}$
$\textbf{7. There are limits to the range of impedances}$
$\textbf{that it can match.**}$

\subsection{Key Points Pi-Pad Networks}  
$\text{1. The }\pi\text{-network’s primary application is to match}$
$\text{a high impedance source to lower value load impedance}$  
$\text{2. It can also be used in reverse to match a low impedance}$
$\text{to a higher impedance.}$  
$\text{3. The }\pi\text{-network also can be considered two back-}$  
$\text{to-back L-networks with a virtual impedance between them.}$  
$\text{4. To use the L-network procedures, you need to assume an}$
$\text{intermediate virtual load/source resistance } R_V = \frac{R_H}{Q^2 + 1}$
$\text{5. $R_H$ is the higher of the two design impedances } R_{SRC} \text{ and } R_L$
$\textbf{6. The resulting } R_V \textbf{ will be lower than either } R_{SRC}$
$\textbf{or } R_L \textbf{ depending on the desired } Q.$
$\textbf{7. Typical }Q \textbf{ values are usually in the 5 to 20 range.}$

\subsection{Key Points T-Pad Networks}  
$\text{1. The main reason to employ a T-network}$
$\text{like the }\pi \text{-network, is to get control of the circuit Q.}$  
$\text{2. Like the }\pi \text{-network, it's used when you need to limit}$
$\text{the bandwidth to reduce harmonics or help filter out}$
$\text{adjacent signals without the use of additional filters.}$   
$\text{3. The T-network also can be considered two cascaded}$  
$\text{L-networks. }\textbf{This is equivalent to the }\pi\textbf{-network above.}$  

\subsection{Note**}

$\text{There is no specific formulas for low pass or}$
$\text{high pass L-pads or T-pads, or }\pi \text{-pads}$ 
$\text{this depends on which is higher, } R_L \text{ or } R_S.$ 

\subsection{L-Pad Formulas (Low Pass 1)**}

$$Q = \sqrt{\frac{R_L}{R_{SRC}} - 1}$$  
$$X_L = QR_{SRC}$$  
$$L = \frac{X_L}{2\pi f}$$  
$$X_C = \frac{R_L}{Q}$$  
$$C = \frac{1}{2\pi f X_C}$$  
$$BW = \frac{f}{Q}$$

\subsection{L-Pad Formulas (Conversion Series-Parallel)}

$$R_S = \text{ series resistance} = \frac{R_p}{Q^2 + 1}$$  
$$R_p = \text{ parallel resistance} = R_S(Q^2 + 1)$$  
$$X_s = \text{ series reactance} = \frac{X_pQ^2}{Q^2 + 1}$$  
$$X_p = \text{ parallel reactance} = \frac{X_S(Q^2 +1)}{Q^2}$$  
$$Q = \sqrt{\frac{R_p}{R_S} - 1} = \frac{X_L}{R_S} = \frac{R_p}{X_C}$$
$$\text{If } Q > 5, y\text{ you can use the simplified approximations:}$$  
$$R_p = Q^2R_S$$  
$$X_p = X_S$$ 

\subsection{L-Pad Formulas (Low Pass 2)**}

$$Q = \sqrt{\frac{R_{SRC}}{R_L} - 1}$$  
$$X_L = QR_{L}$$  
$$L = \frac{X_L}{2\pi f}$$  
$$X_C = \frac{R_{SRC}}{Q}$$  
$$C = \frac{1}{2\pi f X_C}$$  
$$BW = \frac{f}{Q}$$  
$$R_R = \frac{L}{CR} = R(Q^2 +1) = \text{ resonant equivalent resistance}$$  

\subsection{Pi-Pad Formulas (Low Pass)**}

$$BW = \frac{f}{Q}$$  
$$R_V = \frac{R_H}{Q^2 + 1}$$
$$X_L = QR_{L}; \text{ use this formula for } X_{L_1} \text{ and } X_{L_2}$$  
$$L = \frac{X_L}{2\pi f}; \text{ use this formula for } L_1 \text{ and } L_2$$ 
$$X_{C_1} = \frac{R_{SRC}}{Q}$$  
$$X_{C_2} = \frac{R_{L}}{Q}$$  
$$C = \frac{1}{2\pi f X_{C}}; \text{ use this formula for } C_1 \text{ and } C_2$$  
$$Q = \sqrt{\frac{R_{SRC}}{R_L} - 1}; R_{SRC} = R_V$$  
$$\text{Add } L_1 + L_2 \text{ to get total inductance.}$$  

\subsection{T-Pad Formulas (LCC Method)}

$\text{1. Select the desired bandwidth and calculate Q.}$ 
$\text{2. Calculate: } X_L = QR_{SRC}$
$\text{3. Calculate } X_{C_2} = R_L\sqrt{R_{SRC}\frac{Q^2 + 1}{R_L} - 1}$
$\text{4. Calculate } X_{C_1} = R_{SRC}\frac{Q^2 + 1}{Q}\frac{QR_L}{QR_L - X_{C_2}}$
$\text{5. Calculate the inductance } L = \frac{X_L}{2 \pi f}$
$\text{6. Calculate the capacitances } C = \frac{1}{2 \pi f X_C}$

\section{Telephone Set and Loop Subscriber Interface}

$\text{Number of Interconnecting Lines } = \frac{n(n - 1)}{2}$
$\text{Numerical Value of the Wire Pair } = \text{ Tip Color } + \text{ Ring Color}$
$\text{Wire pair No. } = (\text{Binder No.} - 1) * 25 + \text{ Wire pair No. in a 25-pair binder}$

\subsection{Telephone Circuit}

\begin{enumerate}
  \item Dialing Circuit Pulse/DTMF are used to make calls
  \item On-hook/off-hok circuit
  \begin{itemize}
    \item Off-Hook - Telephone is picked-up, off the hook
    \item On-Hook - Telephone is placed on the hook, "hanged up"
  \end{itemize}
  \item Subscriber loop - how power is delivered to the handset (derived from -48 V (DC) from the central office)
  \begin{itemize}
    \item Most subscriber loops are two-wire pairs
  \end{itemize}
  \item Hybrid Circuit - splits the two-wire pairs into four wire: two for transmitting signals, two for receiving signals.
  \begin{itemize}
    \item Full Duplex is made possible
  \end{itemize}
  \item Equalizers - they compensate for different wire lengths from the central office and subcribers. They regulate voice amplitudes 
\end{enumerate}

\subsection{Telephone Ringer}

\begin{enumerate}
  \item Rings to alert the receiver of an incoming call
  \item The ringer signal (90 $V_{rms}$ @ 20 Hz; for US Companies) is superimposed of the -48 V (DC) from the central office
  \item The ringer in a telephone is composed of two bells which are struck during alternative parts of the cycle.
\end{enumerate}

\subsection{Telephone Hybrid}

\begin{enumerate}
  \item Most subscriber loops are two-wire pairs
  \item The Hybrid Circuit splits the two-wire pairs into four wire: two for transmitting signals, two for receiving signals.
  \item Sidetone is the small feedback that allows the user to here him/her-self and adjust their volume accordingly. This is done through a balancing network in the Hybrid Circuit.
\end{enumerate}

\subsection{Dual Tone Multifrequency}

\begin{enumerate}
  \item There's a key pad instead of a rotary for dialing. There are 10 digits (0-9), 2 special characters ( star, and pound), and an optional extra 4 buttons for special functionality ("A", "B", "C", "D"). 
  \item Altogether they form a  "4 X 4" frequency matrix consisting of low-band and high-band frequencies.
  \item The frequency of each button are seperated with a difference of about 10%.
  \item The frequency between low-band and high-band frequencies are 25% apart.
  \item When a button is pressed two frequences are sent to the phone company, a low-band and high-band frequency.
  \item The frequencies, both low-band and high-band, for each number are unique (their harmonics are unique, too) and don't conflict with each other.
  \item The major advantage of DTMF (Touch Tone) over rotary is speed and control.
\end{enumerate}

\subsection{Centralized Switching}

\begin{enumerate}
  \item Centralized Switching rectifies the problem of having all phones in an area interconnecting. This is done by having a temporary connection between the two parties that want to communicate.
\end{enumerate}

\subsection{Local Loop}

\begin{enumerate}
  \item In order for phones to be useful they have to be connected to other phones to form a communications link. Check 
\end{enumerate}

\subsection{Telephone Cable Color Codes}

\begin{enumerate}
  \item Wires are colored coded according to a standard for troubleshooting ease. 
  \item Ten colors are used to identify tip (5 for tip) and ring (5 for ring) wire pairs. Check Numerical Value of the Wire Pair and Wire pair No. equation above.
  \begin{enumerate}
    \item Tip Colors: White = 0; Red = 5; Black = 10; Yellow = 15; Violet = 20
    \item Ring Colors: Blue = 1; Orange = 2; Green = 3; Brown = 4; Slate = 5
  \end{enumerate}
\end{enumerate}

\section{Resonance}

\begin{itemize}
  \item Below steady state solution where voltage leads current, transient phenomenon dies out, when $\phi$ is positive: 
\end{itemize}

$$I = \frac{V_O}{\sqrt{R^2 + (\omega L - \frac{1}{\omega C})^2}} \cos(\omega t - \phi) \tan \phi = \frac{\omega L - \frac{1}{\omega C}}{R}$$  
$$\text{Let } \omega L - \frac{1}{\omega C} = X = \text{reactance}$$  
$$\text{Let }\sqrt{R^2 + (\omega L - \frac{1}{\omega C})^2}= \sqrt{R^2 + X^2} = Z = \text{impedance}$$

$$I = \frac{V_O}{Z} \cos(\omega t - \phi) \tan \phi = \frac{X}{R}$$ 

\begin{itemize}
  \item The Transient solution occurs when current leads voltage $\phi$ is negative. The above doesn't hold yet.
\end{itemize}

$$\text{Let }\frac{V_O}{Z} = I_{max} $$

\begin{itemize}
  \item Resonance occurs when $X = 0$; $\omega L = \frac{1}{\omega C}$; $\omega_O = \frac{1}{\sqrt{LC}}$, $Z = R$; and $\phi = 0$ causing $I_{max} = \frac{V_O}{R}$
\end{itemize}

\begin{itemize}
  \item Width of the of the resonant curve (similar to bandwidth) $\Delta W = \frac{R}{L} = 0.707 I_{max}$
\end{itemize}

\begin{itemize}
  \item Quality Factor $Q = \frac{1}{R}\sqrt{\frac{L}{C}}$
\end{itemize}

\subsection{PB Filter Series}

$$X_L = X_C, \text{ at resonance}$$   
$$V_{O(max)} = \frac{R}{R + R_l} * V_i$$  
$$f_s = \frac{1}{2\pi \sqrt{LC}}$$  
$$Q_s = \frac{X_L}{R + R_l}$$  
$$BW = \frac{f_s}{Q_s}$$  

\subsection{PB Filter Parallel}

$$Z_{T_p} = \text{ is a maximum value at resonance} = Q_l^2R_l$$  
$$V_{O(max)} = \frac{Z_{T_p}}{R + Z_{T_p}} * V_i$$  
$$f_p = \frac{1}{2\pi \sqrt{LC}}$$  
$$Q_p = \frac{X_L}{R_l}$$  
$$BW = \frac{f_p}{Q_p}$$  

\subsection{SB Filter Series}

$$X_L = X_C, \text{ at resonance}$$   
$$V_{O(min)} = \frac{R_l}{R + R_l} * V_i$$  
$$f_s = \frac{1}{2\pi \sqrt{LC}}$$  
$$Q_s = \frac{X_L}{R + R_l}$$  
$$BW = \frac{f_s}{Q_s}$$  

\subsection{SB Filter Parallel}

$$Z_{T_p} = \text{ is a maximum value at resonance} = Q_l^2R_l$$  
$$V_{O(min)} = \frac{R}{R + Z_{T_p}} * V_i$$  
$$f_p = \frac{1}{2\pi \sqrt{LC}}$$  
$$Q_p = \frac{X_L}{R_l}$$  
$$BW = \frac{f_p}{Q_p}$$  

% You can even have references
\rule{0.3\linewidth}{0.25pt}
\scriptsize
\bibliographystyle{abstract}
\bibliography{refFile}
\end{multicols}
\end{document}