\documentclass[10pt,landscape]{article}
\usepackage{multicol}
\usepackage{calc}
\usepackage{ifthen}
\usepackage[landscape]{geometry}
\usepackage{amsmath,amsthm,amsfonts,amssymb}
\usepackage{color,graphicx,overpic}
\usepackage{hyperref}


\pdfinfo{
  /Title (example.pdf)
  /Creator (TeX)
  /Producer (pdfTeX 1.40.0)
  /Author (Seamus)
  /Subject (Example)
  /Keywords (pdflatex, latex,pdftex,tex)}

% This sets page margins to .5 inch if using letter paper, and to 1cm
% if using A4 paper. (This probably isn't strictly necessary.)
% If using another size paper, use default 1cm margins.
\ifthenelse{\lengthtest { \paperwidth = 11in}}
    { \geometry{top=.5in,left=.5in,right=.5in,bottom=.5in} }
    {\ifthenelse{ \lengthtest{ \paperwidth = 297mm}}
        {\geometry{top=1cm,left=1cm,right=1cm,bottom=1cm} }
        {\geometry{top=1cm,left=1cm,right=1cm,bottom=1cm} }
    }

% Turn off header and footer
\pagestyle{empty}

% Redefine section commands to use less space
\makeatletter
\renewcommand{\section}{\@startsection{section}{1}{0mm}%
                                {-1ex plus -.5ex minus -.2ex}%
                                {0.5ex plus .2ex}%x
                                {\normalfont\large\bfseries}}
\renewcommand{\subsection}{\@startsection{subsection}{2}{0mm}%
                                {-1explus -.5ex minus -.2ex}%
                                {0.5ex plus .2ex}%
                                {\normalfont\normalsize\bfseries}}
\renewcommand{\subsubsection}{\@startsection{subsubsection}{3}{0mm}%
                                {-1ex plus -.5ex minus -.2ex}%
                                {1ex plus .2ex}%
                                {\normalfont\small\bfseries}}
\makeatother

% Define BibTeX command
\def\BibTeX{{\rm B\kern-.05em{\sc i\kern-.025em b}\kern-.08em
    T\kern-.1667em\lower.7ex\hbox{E}\kern-.125emX}}

% Don't print section numbers
\setcounter{secnumdepth}{0}


\setlength{\parindent}{0pt}
\setlength{\parskip}{0pt plus 0.5ex}

%My Environments
\newtheorem{example}[section]{Example}
% -----------------------------------------------------------------------

\begin{document}
\raggedright
\footnotesize
\begin{multicols}{3}


% multicol parameters
% These lengths are set only within the two main columns
%\setlength{\columnseprule}{0.25pt}
\setlength{\premulticols}{1pt}
\setlength{\postmulticols}{1pt}
\setlength{\multicolsep}{1pt}
\setlength{\columnsep}{2pt}

\begin{center}
     \Large{\underline{TCET Formula Sheet}} \\
\end{center}

\section{Decibel Formulas}
\subsection{Relative Power Gain}

$$A_P = \frac{P_O}{P_I}$$
\text{where } $P_O$ \text{ and } $P_I$ \text{ are defined as the following:}
$$P_I = \frac{V_I^2}{R_I}$$
$$P_O = \frac{V_O^2}{R_O}$$

\subsection{Relative Voltage Gain}

$$A_V = \frac{V_O}{V_I}$$

\subsection{Relative Power Gain in dB}

$$A_P(db) = 10 \log_{10}A_P$$
$$\text{Given that } R_O = R_I$$  
\text{ If } $R_O \neq R_I$ \text{ then the general form is given by the following:}
$$A_P(db) = 10 \log_{10}\big(\frac{\frac{V_O^2}{R_O}}{\frac{V_I^2}{R_I}}\big)$$

\subsection{Relative Voltage Gain in dB}

$$A_V(db) = 20 \log_{10}\big(\frac{V_O}{V_I}\big) = 20 \log_{10} A_V$$
$$\text{ If } R_O \neq R_I \text{ then the general form is given by the following:}$$
$$A_V(db) = 20 \log_{10}\big(\frac{V_O}{V_I}\big) - 10 \log_{10}\big(\frac{R_O}{R_I}\big)$$

\subsection{Absolute Power Gain dBm}

$$A_{P(dBm)} = 10 \log_{10} \big(\frac{P}{1 \text{ mW}}\big), \text{ dBm}$$

\subsection{Absolute Power Gain dBw}

$$A_{P(dBw)} = 10 \log_{10} \big(\frac{P}{1 \text{ W}}\big), \text{ dBw}$$

\subsection{Signal-to-Noise Ratio}

$$\text{SNR } = 10 \log_{10}\big(\frac{\text{Signal Power}}{\text{Noise Power}}\big)$$ 
$$\text{And given that } R_O = R_I,$$
$$\text{SNR } = 10 \log_{10} \big(\frac{V_S^2}{V_N^2}\big)$$
$$\text{SNR } = 20 \log_{10} \big(\frac{V_S}{V_N}\big)\text{dB}$$

\subsection{Impulse Noise}

$$ dB_S = 20 \log_{10} \big(\frac{P}{0.0002 \space \bar{\mu}}\big) \text{, where P is sound pressure in } \bar{\mu}$$
$$ \bar{\mu} = 1 \space \frac{\text{dyne}}{\text{cm}^2} = 10^{-6} \text{ of atmospheric pressure at sea level}$$

\section{Impedance Matching}

$$R_S = R_L, \text{ for DC sources.}$$
$$Z_S = Z_L, \text{ for AC sources.}$$
$$\text{Source Resistance = Load Resistance}$$
$$P_I = P_O$$  

\subsection{Key Points L-Pad Networks}  
$\text{1. The primary applications of L-networks involve impedance}$
$\text{matching in RF circuits, transmitters, and receivers.}$  
$\text{2. L-networks are useful in matching one amplifier}$
$\text{output to the input of a following stage.}$  
$\text{3. Any RF circuit application covering a narrow frequency}$  
$\text{range is a candidate for an L-network.}$  
$\text{4. There are four basic versions of the L-network,}$
$\text{with two low-pass versions and two high-pass versions.}$
$\text{5. Most widely used since they attenuate harmonics,}$
$\text{noise, and other undesired signals.}$
$\textbf{6. The impedances that are being matched}$
$\textbf{determine the Q (quality factor) of the circuit,}$
$\textbf{which cannot be specified or controlled.**}$
$\textbf{7. There are limits to the range of impedances}$
$\textbf{that it can match.**}$

\subsection{Key Points Pi-Pad Networks}  
$\text{1. The primary applications of L-networks involve impedance}$
$\text{matching in RF circuits, transmitters, and receivers.}$  
$\text{2. L-networks are useful in matching one amplifier}$
$\text{output to the input of a following stage.}$  
$\text{3. Any RF circuit application covering a narrow frequency}$  
$\text{range is a candidate for an L-network.}$  
$\text{4. There are four basic versions of the L-network,}$
$\text{with two low-pass versions and two high-pass versions.}$
$\text{5. Most widely used since they attenuate harmonics,}$
$\text{noise, and other undesired signals.}$
$\textbf{6. The impedances that are being matched}$
$\textbf{determine the Q (quality factor) of the circuit,}$
$\textbf{which cannot be specified or controlled.**}$
$\textbf{7. There are limits to the range of impedances}$
$\textbf{that it can match.**}$

\subsection{Key Points T-Pad Networks}  
$\text{1. The primary applications of L-networks involve impedance}$
$\text{matching in RF circuits, transmitters, and receivers.}$  
$\text{2. L-networks are useful in matching one amplifier}$
$\text{output to the input of a following stage.}$  
$\text{3. Any RF circuit application covering a narrow frequency}$  
$\text{range is a candidate for an L-network.}$  
$\text{4. There are four basic versions of the L-network,}$
$\text{with two low-pass versions and two high-pass versions.}$
$\text{5. Most widely used since they attenuate harmonics,}$
$\text{noise, and other undesired signals.}$
$\textbf{6. The impedances that are being matched}$
$\textbf{determine the Q (quality factor) of the circuit,}$
$\textbf{which cannot be specified or controlled.**}$
$\textbf{7. There are limits to the range of impedances}$
$\textbf{that it can match.**}$


\subsection{L-Pad Formulas (Low Pass 1)}

$$Q = \sqrt{\frac{R_L}{R_S} - 1}$$  
$$X_L = QR_S$$  
$$L = \frac{X_L}{2\pi f}$$  
$$X_C = \frac{R_L}{Q}$$  
$$C = \frac{1}{2\pi f X_C}$$  
$$BW = \frac{f}{Q}$$  

\subsection{L-Pad Formulas (Low Pass 2)}

$$Q = \sqrt{\frac{R_L}{R_S} - 1}$$  
$$X_L = QR_S$$  
$$L = \frac{X_L}{2\pi f}$$  
$$X_C = \frac{R_L}{Q}$$  
$$C = \frac{1}{2\pi f X_C}$$  
$$BW = \frac{f}{Q}$$  

\section{Resonance}
Etc.

\section{Telecommunications (Chapter 11)}
Etc.

% You can even have references
\rule{0.3\linewidth}{0.25pt}
\scriptsize
\bibliographystyle{abstract}
\bibliography{refFile}
\end{multicols}
\end{document}