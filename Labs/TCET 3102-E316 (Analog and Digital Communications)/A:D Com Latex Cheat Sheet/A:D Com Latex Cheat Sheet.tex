\documentclass[10pt,landscape]{article}
\usepackage{multicol}
\usepackage{calc}
\usepackage{ifthen}
\usepackage[landscape]{geometry}
\usepackage{amsmath,amsthm,amsfonts,amssymb}
\usepackage{color,graphicx,overpic}
\usepackage{hyperref}
\usepackage{listings}


\pdfinfo{
  /Title (example.pdf)
  /Creator (TeX)
  /Producer (pdfTeX 1.40.0)
  /Author (Seamus)
  /Subject (Example)
  /Keywords (pdflatex, latex,pdftex,tex)}

% This sets page margins to .5 inch if using letter paper, and to 1cm
% if using A4 paper. (This probably isn't strictly necessary.)
% If using another size paper, use default 1cm margins.
\ifthenelse{\lengthtest { \paperwidth = 11in}}
    { \geometry{top=.2in,left=.2in,right=.2in,bottom=.2in} }
    {\ifthenelse{ \lengthtest{ \paperwidth = 297mm}}
        {\geometry{top=1cm,left=1cm,right=1cm,bottom=1cm} }
        {\geometry{top=1cm,left=1cm,right=1cm,bottom=1cm} }
    }

% Turn off header and footer
\pagestyle{empty}

% Redefine section commands to use less space
\makeatletter
\renewcommand{\section}{\@startsection{section}{1}{0mm}%
                                {-1ex plus -.5ex minus -.2ex}%
                                {0.5ex plus .2ex}%x
                                {\normalfont\large\bfseries}}
\renewcommand{\subsection}{\@startsection{subsection}{2}{0mm}%
                                {-1explus -.5ex minus -.2ex}%
                                {0.5ex plus .2ex}%
                                {\normalfont\normalsize\bfseries}}
\renewcommand{\subsubsection}{\@startsection{subsubsection}{3}{0mm}%
                                {-1ex plus -.5ex minus -.2ex}%
                                {1ex plus .2ex}%
                                {\normalfont\small\bfseries}}
\makeatother

% Define BibTeX command
\def\BibTeX{{\rm B\kern-.05em{\sc i\kern-.025em b}\kern-.08em
    T\kern-.1667em\lower.7ex\hbox{E}\kern-.125emX}}

% Don't print section numbers
\setcounter{secnumdepth}{0}


\setlength{\parindent}{0pt}
\setlength{\parskip}{0pt plus 0.5ex}

%My Environments
\newtheorem{example}[section]{Example}
% -----------------------------------------------------------------------

\begin{document}
\raggedright
\footnotesize
\begin{multicols}{3}


% multicol parameters
% These lengths are set only within the two main columns
%\setlength{\columnseprule}{0.25pt}
\setlength{\premulticols}{1pt}
\setlength{\postmulticols}{1pt}
\setlength{\multicolsep}{1pt}
\setlength{\columnsep}{2pt}

\begin{center}
     \Large{\underline{A/D Com. Formula Sheet}} \\
\end{center}

\section{American Wire Gauge (AWG/SLIC):}
\subsection{Definition:}

\textbf{Subscriber Loop Interface Circuit: }

\begin{enumerate}
  \item \text{Located at the central office.}
  \item \text{Supports various functions defined under} \\ \text{the BORSCHT anagram.}  
  \item \text{This is done through the Subscriber Loop Lines.} \\ \text{These are lines connecting the central} \\ \text{office to the patrons homes.}
\end{enumerate}

\textbf{Objective:}

\begin{enumerate}
  \item \text{To use the largest possible AWG} \\ \text{to support certain requirements.} \\ \text{Basically finding where company and consumer interest} \\ \text{align. Not to expensive for the company while} \\ \text{also providing good service to the consumer.}
\end{enumerate}

\subsection{Method:}

\textbf{Assumptions:}
\begin{enumerate}
  \item \text{Minimum Current: 20} $mA$
  \item \text{Maximum Current: 120} $mA$
  \item \text{Central Office Resistance: 400} $\Omega$
  \item \text{Telephone Resistance: 400} $\Omega$
\end{enumerate}

\subsubsection{Example 1:}

\textbf{Find AWG wire to support the following:}
\begin{enumerate}
  \item \text{25} $mA$ \text{ current.}
  \item \text{Distance: 5} $km$
  \item \text{Max Attenuation Unloaded: 7} $dB$
\end{enumerate}

\subsubsection{Answer:}

\textbf{Find Required DC Resistance:}
$$V = IR$$
$$48 \space V = 25 \space mA  \big(400 + 400 + x\big) \space \Omega$$
$$x \space \Omega = \Big(\frac{48 \space V}{25 \space mA}\Big) - 800 \space \Omega$$  
$$x \space \Omega = 1120 \space \Omega$$


\textbf{If the following is true then that is the AWG to use:}

\begin{itemize}
  \item $\text{miles} * \text{Attenuation (dB/mile unloaded)} < 7 \space dB$
  \item $\text{miles} * \text{Round Trip Loop Resistance} < 1120 \space \Omega$
\end{itemize}


\subsubsection{Example 2:}

\textbf{Goal: Support the following system}
\begin{enumerate}
  \item \text{24} $mA$ \text{ DC minimum current.}
  \item $400 \Omega$ \text{ telephone and CO Resistances.} 
  \item \text{Attenuation:} $< 7 \space dB$
  \item \text{Using AWG 19 wire}
\end{enumerate}

\subsubsection{Answer:}

\textbf{Find Required DC Resistance:}
$$V = IR$$
$$48 \space V = 24 \space mA  \big(400 + 400 + x\big) \space \Omega$$
$$x \space \Omega = \Big(\frac{48 \space V}{24 \space mA}\Big) - 800 \space \Omega$$  
$$x \space \Omega = 1200 \space \Omega$$

\textbf{Look at table for AWG 19: Properties}
\begin{itemize}
  \item \text{Round Trip Loop Resistance:} $\frac{85 \Omega}{mi}$
  \item \text{Attenuation (dB/mile unloaded):} $\frac{1.12 \text{ dB}}{mi}$
\end{itemize}


\textbf{Find Resistance Constraint:}
$$ x \space mi = \frac{1200 \Omega}{80 \frac{\Omega}{mi}} = 14.1 \space mi = 22.7 km$$

\textbf{Find Attenuation Constraint:}
$$ x \space mi = \frac{9 \space dB}{1.12 \frac{dB}{mi}} = 8.04 \space mi = 12.49 km$$

\textbf{Max Distance is the smaller of the two,} \\ \textbf{as both conditions must be satisfied}

\section{Fourier Series}

\subsection{Type 1}

\begin{itemize}
  \item \textbf{Function}
    $$f(x) =
      \begin{cases}
        -1 & \quad - \pi < x < 0 \\
        1 & \quad  0 < x < \pi 
      \end{cases}
    $$
  \item \textbf{Fourier Series Equations}
    $$a_0 = \frac{1}{2 \pi}\int\limits_{-\pi}^{\pi} f(x) dx$$
    $$a_n = \frac{1}{ \pi}\int\limits_{-\pi}^{\pi} f(x) \cos(nx) dx$$  
    $$b_n = \frac{1}{ \pi}\int\limits_{-\pi}^{\pi} fx() \sin(nx) dx$$ 

  \item \text{"U" substitution: Let} $u = nx$; $\frac{du}{dx} = n \space dx$  

\end{itemize}

\subsection{Type 2}

\begin{itemize}
  \item \textbf{Function}
    $$f(x) =
      \begin{cases}
        -1 & \quad - \frac{1}{2} T < x < 0 \\
        1 & \quad  0 < x < \frac{1}{2} T 
      \end{cases}
    $$
  \item \textbf{Generalized Fourier Series Equations}
    $$a_0 = \frac{1}{2L} \int\limits_{-L}^{L} f(x) dx$$  
    $$a_n = \frac{1}{L} \int\limits_{-L}^{L} f(x) \cos(\frac{n \pi x}{L})dx$$  
    $$b_n = \frac{1}{L} \int\limits_{-L}^{L} f(x) \sin(\frac{n \pi x}{L})dx$$ 

   \item \text{"U" substitution: Let} $1 = \cfrac{\frac{2n \pi}{T}}{\frac{2n \pi}{T}}$, $u = \frac{2n \pi x}{T}$, $du = \frac{2n \pi}{T}$
    
\end{itemize} 

\section{Fourier Transforms}

\subsection{Type 1}

\begin{itemize}
  \item \textbf{Function}
    $$f(x) =
      \begin{cases}
        -1 & \quad - \pi < x < 0 \\
        1 & \quad  0 < x < \pi 
      \end{cases}
    $$
  \item \textbf{Fourier Series Equations}
    $$c_0 = a_0 = \frac{1}{2 \pi}\int\limits_{-\pi}^{\pi} f(x) dx$$
    $$c_n = \frac{a_n - ib_n}{2} = \frac{1}{2 \pi}\int\limits_{-\pi}^{\pi} f(x)e^{-inx} dx$$  
    $$c_{-n} = \frac{a_n + ib_n}{2} = \frac{1}{2 \pi}\int\limits_{-\pi}^{\pi} f(x)e^{inx} dx$$  
  \item \text{"U" substitution: Let} $u = nx$; $\frac{du}{dx} = n \space dx$  

\end{itemize}

\section{Finding Voltage Spectral Densities}

$$ VPD = A\tau*\text{sinc}(f\tau) = A\tau*\frac{\sin{f \tau \pi}}{f \tau \pi}$$

\begin{lstlisting}
# Pulse width (Tau) = 3 microseconds ()
# Amplitude (Amp) = 5 Volts
# Frequency (Freq) = 30, 100, 3000
# Voltage Spectral Density = 
# Amp * Tau * (sin(freq*Tau*pi)/(freq*Tau*pi))

def voltageSpectralDensity(freq, 
                           Tau= 3*m.pow(10, -3), 
                           Amp= 5):
    """Calculates the Voltage Spectral Density 
    given Pulse Width (Tau), Amplitude (Amp), 
    Frequency."""
    sinc = lambda K: m.sin(K*m.pi)/(K*m.pi)
    return Amp*Tau*sinc(freq*Tau)
\end{lstlisting}

\section{One of the HW Questions}

\text{A sine wave is described by 5sin(300t+27°), where t} \\ \text{is the time in seconds. Determine the waveform (a) amplitude,} \\ \text{(b) rms value, (3) frequency, (4) periodic time, and} \\ \text{(5) time lag or lead.}
    $$\space$$
\begin{enumerate}
    \item \text{Amplitude: } $5 \text{ Volts}$
    \item \text{RMS value: } $\cfrac{1}{\sqrt{2}} * 5 \approx 0.7071 * 5 \approx 3.53 \text{ Volts}$
    \item \text{Frequency: } $\omega = 2 \pi f = 300$ so $f = \cfrac{300}{2 \pi} \approx 47.75$
    \item \text{Period: } $T = \cfrac{1}{f} \approx \cfrac{1}{47.75} \approx 0.021$
    \item \text{Time: Leading by } $27^\circ$
\end{enumerate}

\section{Bode Plot}

\subsection{Method:}

\textbf{We are given a transfer function} $$H(s) = \cfrac{K(s+z_1)}{s(s+p_1)}$$
\textbf{where:}  

\begin{itemize}
  \item $K = \text{constant}$ 
  \item $z_1 = \text{the zeros of the transfer function}$
  \item $p_1 = \text{the poles of the transfer function}$
\end{itemize}

\textbf{We rewrite it by factoring both the numerator and} \\ \textbf{denominator into the standard form, giving us} $$H(s) = \cfrac{K z_1(\frac{s}{z_1} + 1)}{s p_1(\frac{s}{p_1} + 1)}$$

\textbf{Then we do the following:}

\begin{itemize}
  \item \text{Find K (dB) given K } $K_{dB} = 20 \log_{10} (K)$
  \item \text{Find the value of the zeros } $z_n; \text{ where } n = 1, 2, 3,...$ 
  \item \text{Find the value of the poles } $p_n; \text{ where } n = 1, 2, 3,...$ 
\end{itemize}

\textbf{To plot the bode plot we follow these instructions}

\begin{itemize}
  \item \text{Effect of Constant Terms: Constant terms such as K} \\ \text{contribute a straight horizontal line of magnitude 20 log10(K)}
  \item \text{Effect of Individual Zeros: Each occurrence} \\ \text{of this causes a positively sloped line passing through the} \\ \text{ value of 'z' with a rise of 20 db over a decade.}
  \item \text{Effect of Individual Poles: Each occurrence} \\ \text{of this causes a negatively sloped line passing through the} \\ \text{ value of 'p' with a rise of 20 db over a decade.}
\end{itemize}

\section{Transfer Function:}

\textbf{Transfer Function for Low Pass Filters } $$|H(j\omega)| = \Big|\frac{V_{OUT}(j\omega)}{V_{IN}(j\omega)}\Big|  = \frac{1}{\sqrt{1 + \left(\frac{\omega}{\omega_C}\right)^{2n}}} = \frac{1}{\sqrt{1 + \left(\frac{f}{f_C}\right)^{2n}}}$$

\textbf{where } $f_C = \frac{1}{2 \pi R C} \space Hz$

\section{Convolution}

\begin{itemize}
  \item \text{Convolution of two rectangular pulses of unequal} \\ \text{duration will be a trapezoid}
  \item \text{Convolution of two rectangular pulses of equal} \\ \text{duration will be a triangle}
\end{itemize}

$$y(t) = x(t) * h(t) = \int_{-\infty}^{\infty} x(\tau)h(t-\tau) d \tau$$

\subsection{Official Steps}

\begin{enumerate}
  \item \text{Given the function }  $$y(t) = \int_{-\infty}^{\infty} x(\tau)h(t-\tau) d \tau$$ \\ \text{we take the wave representing } $$h(t-\tau)$$ \\ \text{and flip it across the y-axis. We then start the wave at the } \\ \text{point on the x-axis closest to } $-\infty$.  
  \item \text{The magnitudes of } $x(\tau)$ and $h(t-\tau)$ \text{ is the amplitude} \\ \text{of the waves}
\end{enumerate}

\subsection{Unofficial Idea/Steps}

\begin{itemize}
  \item \text{basically we use the above function for output.} \\ \text{we only calculate the integral when both square waves} \\ \text{intersect.}
  \item \text{The magnitudes of } $x(\tau)$ and $h(t-\tau)$ \text{ is the amplitude} \\ \text{of the waves}
\end{itemize}

% You can even have references
\rule{0.3\linewidth}{0.25pt}
\scriptsize
\bibliographystyle{abstract}
\bibliography{refFile}
\end{multicols}
\end{document}