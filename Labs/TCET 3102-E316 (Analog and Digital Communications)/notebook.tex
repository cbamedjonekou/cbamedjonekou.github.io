
% Default to the notebook output style

    


% Inherit from the specified cell style.




    
\documentclass[11pt]{article}

    
    
    \usepackage[T1]{fontenc}
    % Nicer default font (+ math font) than Computer Modern for most use cases
    \usepackage{mathpazo}

    % Basic figure setup, for now with no caption control since it's done
    % automatically by Pandoc (which extracts ![](path) syntax from Markdown).
    \usepackage{graphicx}
    % We will generate all images so they have a width \maxwidth. This means
    % that they will get their normal width if they fit onto the page, but
    % are scaled down if they would overflow the margins.
    \makeatletter
    \def\maxwidth{\ifdim\Gin@nat@width>\linewidth\linewidth
    \else\Gin@nat@width\fi}
    \makeatother
    \let\Oldincludegraphics\includegraphics
    % Set max figure width to be 80% of text width, for now hardcoded.
    \renewcommand{\includegraphics}[1]{\Oldincludegraphics[width=.8\maxwidth]{#1}}
    % Ensure that by default, figures have no caption (until we provide a
    % proper Figure object with a Caption API and a way to capture that
    % in the conversion process - todo).
    \usepackage{caption}
    \DeclareCaptionLabelFormat{nolabel}{}
    \captionsetup{labelformat=nolabel}

    \usepackage{adjustbox} % Used to constrain images to a maximum size 
    \usepackage{xcolor} % Allow colors to be defined
    \usepackage{enumerate} % Needed for markdown enumerations to work
    \usepackage{geometry} % Used to adjust the document margins
    \usepackage{amsmath} % Equations
    \usepackage{amssymb} % Equations
    \usepackage{textcomp} % defines textquotesingle
    % Hack from http://tex.stackexchange.com/a/47451/13684:
    \AtBeginDocument{%
        \def\PYZsq{\textquotesingle}% Upright quotes in Pygmentized code
    }
    \usepackage{upquote} % Upright quotes for verbatim code
    \usepackage{eurosym} % defines \euro
    \usepackage[mathletters]{ucs} % Extended unicode (utf-8) support
    \usepackage[utf8x]{inputenc} % Allow utf-8 characters in the tex document
    \usepackage{fancyvrb} % verbatim replacement that allows latex
    \usepackage{grffile} % extends the file name processing of package graphics 
                         % to support a larger range 
    % The hyperref package gives us a pdf with properly built
    % internal navigation ('pdf bookmarks' for the table of contents,
    % internal cross-reference links, web links for URLs, etc.)
    \usepackage{hyperref}
    \usepackage{longtable} % longtable support required by pandoc >1.10
    \usepackage{booktabs}  % table support for pandoc > 1.12.2
    \usepackage[inline]{enumitem} % IRkernel/repr support (it uses the enumerate* environment)
    \usepackage[normalem]{ulem} % ulem is needed to support strikethroughs (\sout)
                                % normalem makes italics be italics, not underlines
    

    
    
    % Colors for the hyperref package
    \definecolor{urlcolor}{rgb}{0,.145,.698}
    \definecolor{linkcolor}{rgb}{.71,0.21,0.01}
    \definecolor{citecolor}{rgb}{.12,.54,.11}

    % ANSI colors
    \definecolor{ansi-black}{HTML}{3E424D}
    \definecolor{ansi-black-intense}{HTML}{282C36}
    \definecolor{ansi-red}{HTML}{E75C58}
    \definecolor{ansi-red-intense}{HTML}{B22B31}
    \definecolor{ansi-green}{HTML}{00A250}
    \definecolor{ansi-green-intense}{HTML}{007427}
    \definecolor{ansi-yellow}{HTML}{DDB62B}
    \definecolor{ansi-yellow-intense}{HTML}{B27D12}
    \definecolor{ansi-blue}{HTML}{208FFB}
    \definecolor{ansi-blue-intense}{HTML}{0065CA}
    \definecolor{ansi-magenta}{HTML}{D160C4}
    \definecolor{ansi-magenta-intense}{HTML}{A03196}
    \definecolor{ansi-cyan}{HTML}{60C6C8}
    \definecolor{ansi-cyan-intense}{HTML}{258F8F}
    \definecolor{ansi-white}{HTML}{C5C1B4}
    \definecolor{ansi-white-intense}{HTML}{A1A6B2}

    % commands and environments needed by pandoc snippets
    % extracted from the output of `pandoc -s`
    \providecommand{\tightlist}{%
      \setlength{\itemsep}{0pt}\setlength{\parskip}{0pt}}
    \DefineVerbatimEnvironment{Highlighting}{Verbatim}{commandchars=\\\{\}}
    % Add ',fontsize=\small' for more characters per line
    \newenvironment{Shaded}{}{}
    \newcommand{\KeywordTok}[1]{\textcolor[rgb]{0.00,0.44,0.13}{\textbf{{#1}}}}
    \newcommand{\DataTypeTok}[1]{\textcolor[rgb]{0.56,0.13,0.00}{{#1}}}
    \newcommand{\DecValTok}[1]{\textcolor[rgb]{0.25,0.63,0.44}{{#1}}}
    \newcommand{\BaseNTok}[1]{\textcolor[rgb]{0.25,0.63,0.44}{{#1}}}
    \newcommand{\FloatTok}[1]{\textcolor[rgb]{0.25,0.63,0.44}{{#1}}}
    \newcommand{\CharTok}[1]{\textcolor[rgb]{0.25,0.44,0.63}{{#1}}}
    \newcommand{\StringTok}[1]{\textcolor[rgb]{0.25,0.44,0.63}{{#1}}}
    \newcommand{\CommentTok}[1]{\textcolor[rgb]{0.38,0.63,0.69}{\textit{{#1}}}}
    \newcommand{\OtherTok}[1]{\textcolor[rgb]{0.00,0.44,0.13}{{#1}}}
    \newcommand{\AlertTok}[1]{\textcolor[rgb]{1.00,0.00,0.00}{\textbf{{#1}}}}
    \newcommand{\FunctionTok}[1]{\textcolor[rgb]{0.02,0.16,0.49}{{#1}}}
    \newcommand{\RegionMarkerTok}[1]{{#1}}
    \newcommand{\ErrorTok}[1]{\textcolor[rgb]{1.00,0.00,0.00}{\textbf{{#1}}}}
    \newcommand{\NormalTok}[1]{{#1}}
    
    % Additional commands for more recent versions of Pandoc
    \newcommand{\ConstantTok}[1]{\textcolor[rgb]{0.53,0.00,0.00}{{#1}}}
    \newcommand{\SpecialCharTok}[1]{\textcolor[rgb]{0.25,0.44,0.63}{{#1}}}
    \newcommand{\VerbatimStringTok}[1]{\textcolor[rgb]{0.25,0.44,0.63}{{#1}}}
    \newcommand{\SpecialStringTok}[1]{\textcolor[rgb]{0.73,0.40,0.53}{{#1}}}
    \newcommand{\ImportTok}[1]{{#1}}
    \newcommand{\DocumentationTok}[1]{\textcolor[rgb]{0.73,0.13,0.13}{\textit{{#1}}}}
    \newcommand{\AnnotationTok}[1]{\textcolor[rgb]{0.38,0.63,0.69}{\textbf{\textit{{#1}}}}}
    \newcommand{\CommentVarTok}[1]{\textcolor[rgb]{0.38,0.63,0.69}{\textbf{\textit{{#1}}}}}
    \newcommand{\VariableTok}[1]{\textcolor[rgb]{0.10,0.09,0.49}{{#1}}}
    \newcommand{\ControlFlowTok}[1]{\textcolor[rgb]{0.00,0.44,0.13}{\textbf{{#1}}}}
    \newcommand{\OperatorTok}[1]{\textcolor[rgb]{0.40,0.40,0.40}{{#1}}}
    \newcommand{\BuiltInTok}[1]{{#1}}
    \newcommand{\ExtensionTok}[1]{{#1}}
    \newcommand{\PreprocessorTok}[1]{\textcolor[rgb]{0.74,0.48,0.00}{{#1}}}
    \newcommand{\AttributeTok}[1]{\textcolor[rgb]{0.49,0.56,0.16}{{#1}}}
    \newcommand{\InformationTok}[1]{\textcolor[rgb]{0.38,0.63,0.69}{\textbf{\textit{{#1}}}}}
    \newcommand{\WarningTok}[1]{\textcolor[rgb]{0.38,0.63,0.69}{\textbf{\textit{{#1}}}}}
    
    
    % Define a nice break command that doesn't care if a line doesn't already
    % exist.
    \def\br{\hspace*{\fill} \\* }
    % Math Jax compatability definitions
    \def\gt{>}
    \def\lt{<}
    % Document parameters
    \title{TCET 3102-E316 (Analog and Digital Communications) Lab 4}
    
    
    

    % Pygments definitions
    
\makeatletter
\def\PY@reset{\let\PY@it=\relax \let\PY@bf=\relax%
    \let\PY@ul=\relax \let\PY@tc=\relax%
    \let\PY@bc=\relax \let\PY@ff=\relax}
\def\PY@tok#1{\csname PY@tok@#1\endcsname}
\def\PY@toks#1+{\ifx\relax#1\empty\else%
    \PY@tok{#1}\expandafter\PY@toks\fi}
\def\PY@do#1{\PY@bc{\PY@tc{\PY@ul{%
    \PY@it{\PY@bf{\PY@ff{#1}}}}}}}
\def\PY#1#2{\PY@reset\PY@toks#1+\relax+\PY@do{#2}}

\expandafter\def\csname PY@tok@w\endcsname{\def\PY@tc##1{\textcolor[rgb]{0.73,0.73,0.73}{##1}}}
\expandafter\def\csname PY@tok@c\endcsname{\let\PY@it=\textit\def\PY@tc##1{\textcolor[rgb]{0.25,0.50,0.50}{##1}}}
\expandafter\def\csname PY@tok@cp\endcsname{\def\PY@tc##1{\textcolor[rgb]{0.74,0.48,0.00}{##1}}}
\expandafter\def\csname PY@tok@k\endcsname{\let\PY@bf=\textbf\def\PY@tc##1{\textcolor[rgb]{0.00,0.50,0.00}{##1}}}
\expandafter\def\csname PY@tok@kp\endcsname{\def\PY@tc##1{\textcolor[rgb]{0.00,0.50,0.00}{##1}}}
\expandafter\def\csname PY@tok@kt\endcsname{\def\PY@tc##1{\textcolor[rgb]{0.69,0.00,0.25}{##1}}}
\expandafter\def\csname PY@tok@o\endcsname{\def\PY@tc##1{\textcolor[rgb]{0.40,0.40,0.40}{##1}}}
\expandafter\def\csname PY@tok@ow\endcsname{\let\PY@bf=\textbf\def\PY@tc##1{\textcolor[rgb]{0.67,0.13,1.00}{##1}}}
\expandafter\def\csname PY@tok@nb\endcsname{\def\PY@tc##1{\textcolor[rgb]{0.00,0.50,0.00}{##1}}}
\expandafter\def\csname PY@tok@nf\endcsname{\def\PY@tc##1{\textcolor[rgb]{0.00,0.00,1.00}{##1}}}
\expandafter\def\csname PY@tok@nc\endcsname{\let\PY@bf=\textbf\def\PY@tc##1{\textcolor[rgb]{0.00,0.00,1.00}{##1}}}
\expandafter\def\csname PY@tok@nn\endcsname{\let\PY@bf=\textbf\def\PY@tc##1{\textcolor[rgb]{0.00,0.00,1.00}{##1}}}
\expandafter\def\csname PY@tok@ne\endcsname{\let\PY@bf=\textbf\def\PY@tc##1{\textcolor[rgb]{0.82,0.25,0.23}{##1}}}
\expandafter\def\csname PY@tok@nv\endcsname{\def\PY@tc##1{\textcolor[rgb]{0.10,0.09,0.49}{##1}}}
\expandafter\def\csname PY@tok@no\endcsname{\def\PY@tc##1{\textcolor[rgb]{0.53,0.00,0.00}{##1}}}
\expandafter\def\csname PY@tok@nl\endcsname{\def\PY@tc##1{\textcolor[rgb]{0.63,0.63,0.00}{##1}}}
\expandafter\def\csname PY@tok@ni\endcsname{\let\PY@bf=\textbf\def\PY@tc##1{\textcolor[rgb]{0.60,0.60,0.60}{##1}}}
\expandafter\def\csname PY@tok@na\endcsname{\def\PY@tc##1{\textcolor[rgb]{0.49,0.56,0.16}{##1}}}
\expandafter\def\csname PY@tok@nt\endcsname{\let\PY@bf=\textbf\def\PY@tc##1{\textcolor[rgb]{0.00,0.50,0.00}{##1}}}
\expandafter\def\csname PY@tok@nd\endcsname{\def\PY@tc##1{\textcolor[rgb]{0.67,0.13,1.00}{##1}}}
\expandafter\def\csname PY@tok@s\endcsname{\def\PY@tc##1{\textcolor[rgb]{0.73,0.13,0.13}{##1}}}
\expandafter\def\csname PY@tok@sd\endcsname{\let\PY@it=\textit\def\PY@tc##1{\textcolor[rgb]{0.73,0.13,0.13}{##1}}}
\expandafter\def\csname PY@tok@si\endcsname{\let\PY@bf=\textbf\def\PY@tc##1{\textcolor[rgb]{0.73,0.40,0.53}{##1}}}
\expandafter\def\csname PY@tok@se\endcsname{\let\PY@bf=\textbf\def\PY@tc##1{\textcolor[rgb]{0.73,0.40,0.13}{##1}}}
\expandafter\def\csname PY@tok@sr\endcsname{\def\PY@tc##1{\textcolor[rgb]{0.73,0.40,0.53}{##1}}}
\expandafter\def\csname PY@tok@ss\endcsname{\def\PY@tc##1{\textcolor[rgb]{0.10,0.09,0.49}{##1}}}
\expandafter\def\csname PY@tok@sx\endcsname{\def\PY@tc##1{\textcolor[rgb]{0.00,0.50,0.00}{##1}}}
\expandafter\def\csname PY@tok@m\endcsname{\def\PY@tc##1{\textcolor[rgb]{0.40,0.40,0.40}{##1}}}
\expandafter\def\csname PY@tok@gh\endcsname{\let\PY@bf=\textbf\def\PY@tc##1{\textcolor[rgb]{0.00,0.00,0.50}{##1}}}
\expandafter\def\csname PY@tok@gu\endcsname{\let\PY@bf=\textbf\def\PY@tc##1{\textcolor[rgb]{0.50,0.00,0.50}{##1}}}
\expandafter\def\csname PY@tok@gd\endcsname{\def\PY@tc##1{\textcolor[rgb]{0.63,0.00,0.00}{##1}}}
\expandafter\def\csname PY@tok@gi\endcsname{\def\PY@tc##1{\textcolor[rgb]{0.00,0.63,0.00}{##1}}}
\expandafter\def\csname PY@tok@gr\endcsname{\def\PY@tc##1{\textcolor[rgb]{1.00,0.00,0.00}{##1}}}
\expandafter\def\csname PY@tok@ge\endcsname{\let\PY@it=\textit}
\expandafter\def\csname PY@tok@gs\endcsname{\let\PY@bf=\textbf}
\expandafter\def\csname PY@tok@gp\endcsname{\let\PY@bf=\textbf\def\PY@tc##1{\textcolor[rgb]{0.00,0.00,0.50}{##1}}}
\expandafter\def\csname PY@tok@go\endcsname{\def\PY@tc##1{\textcolor[rgb]{0.53,0.53,0.53}{##1}}}
\expandafter\def\csname PY@tok@gt\endcsname{\def\PY@tc##1{\textcolor[rgb]{0.00,0.27,0.87}{##1}}}
\expandafter\def\csname PY@tok@err\endcsname{\def\PY@bc##1{\setlength{\fboxsep}{0pt}\fcolorbox[rgb]{1.00,0.00,0.00}{1,1,1}{\strut ##1}}}
\expandafter\def\csname PY@tok@kc\endcsname{\let\PY@bf=\textbf\def\PY@tc##1{\textcolor[rgb]{0.00,0.50,0.00}{##1}}}
\expandafter\def\csname PY@tok@kd\endcsname{\let\PY@bf=\textbf\def\PY@tc##1{\textcolor[rgb]{0.00,0.50,0.00}{##1}}}
\expandafter\def\csname PY@tok@kn\endcsname{\let\PY@bf=\textbf\def\PY@tc##1{\textcolor[rgb]{0.00,0.50,0.00}{##1}}}
\expandafter\def\csname PY@tok@kr\endcsname{\let\PY@bf=\textbf\def\PY@tc##1{\textcolor[rgb]{0.00,0.50,0.00}{##1}}}
\expandafter\def\csname PY@tok@bp\endcsname{\def\PY@tc##1{\textcolor[rgb]{0.00,0.50,0.00}{##1}}}
\expandafter\def\csname PY@tok@fm\endcsname{\def\PY@tc##1{\textcolor[rgb]{0.00,0.00,1.00}{##1}}}
\expandafter\def\csname PY@tok@vc\endcsname{\def\PY@tc##1{\textcolor[rgb]{0.10,0.09,0.49}{##1}}}
\expandafter\def\csname PY@tok@vg\endcsname{\def\PY@tc##1{\textcolor[rgb]{0.10,0.09,0.49}{##1}}}
\expandafter\def\csname PY@tok@vi\endcsname{\def\PY@tc##1{\textcolor[rgb]{0.10,0.09,0.49}{##1}}}
\expandafter\def\csname PY@tok@vm\endcsname{\def\PY@tc##1{\textcolor[rgb]{0.10,0.09,0.49}{##1}}}
\expandafter\def\csname PY@tok@sa\endcsname{\def\PY@tc##1{\textcolor[rgb]{0.73,0.13,0.13}{##1}}}
\expandafter\def\csname PY@tok@sb\endcsname{\def\PY@tc##1{\textcolor[rgb]{0.73,0.13,0.13}{##1}}}
\expandafter\def\csname PY@tok@sc\endcsname{\def\PY@tc##1{\textcolor[rgb]{0.73,0.13,0.13}{##1}}}
\expandafter\def\csname PY@tok@dl\endcsname{\def\PY@tc##1{\textcolor[rgb]{0.73,0.13,0.13}{##1}}}
\expandafter\def\csname PY@tok@s2\endcsname{\def\PY@tc##1{\textcolor[rgb]{0.73,0.13,0.13}{##1}}}
\expandafter\def\csname PY@tok@sh\endcsname{\def\PY@tc##1{\textcolor[rgb]{0.73,0.13,0.13}{##1}}}
\expandafter\def\csname PY@tok@s1\endcsname{\def\PY@tc##1{\textcolor[rgb]{0.73,0.13,0.13}{##1}}}
\expandafter\def\csname PY@tok@mb\endcsname{\def\PY@tc##1{\textcolor[rgb]{0.40,0.40,0.40}{##1}}}
\expandafter\def\csname PY@tok@mf\endcsname{\def\PY@tc##1{\textcolor[rgb]{0.40,0.40,0.40}{##1}}}
\expandafter\def\csname PY@tok@mh\endcsname{\def\PY@tc##1{\textcolor[rgb]{0.40,0.40,0.40}{##1}}}
\expandafter\def\csname PY@tok@mi\endcsname{\def\PY@tc##1{\textcolor[rgb]{0.40,0.40,0.40}{##1}}}
\expandafter\def\csname PY@tok@il\endcsname{\def\PY@tc##1{\textcolor[rgb]{0.40,0.40,0.40}{##1}}}
\expandafter\def\csname PY@tok@mo\endcsname{\def\PY@tc##1{\textcolor[rgb]{0.40,0.40,0.40}{##1}}}
\expandafter\def\csname PY@tok@ch\endcsname{\let\PY@it=\textit\def\PY@tc##1{\textcolor[rgb]{0.25,0.50,0.50}{##1}}}
\expandafter\def\csname PY@tok@cm\endcsname{\let\PY@it=\textit\def\PY@tc##1{\textcolor[rgb]{0.25,0.50,0.50}{##1}}}
\expandafter\def\csname PY@tok@cpf\endcsname{\let\PY@it=\textit\def\PY@tc##1{\textcolor[rgb]{0.25,0.50,0.50}{##1}}}
\expandafter\def\csname PY@tok@c1\endcsname{\let\PY@it=\textit\def\PY@tc##1{\textcolor[rgb]{0.25,0.50,0.50}{##1}}}
\expandafter\def\csname PY@tok@cs\endcsname{\let\PY@it=\textit\def\PY@tc##1{\textcolor[rgb]{0.25,0.50,0.50}{##1}}}

\def\PYZbs{\char`\\}
\def\PYZus{\char`\_}
\def\PYZob{\char`\{}
\def\PYZcb{\char`\}}
\def\PYZca{\char`\^}
\def\PYZam{\char`\&}
\def\PYZlt{\char`\<}
\def\PYZgt{\char`\>}
\def\PYZsh{\char`\#}
\def\PYZpc{\char`\%}
\def\PYZdl{\char`\$}
\def\PYZhy{\char`\-}
\def\PYZsq{\char`\'}
\def\PYZdq{\char`\"}
\def\PYZti{\char`\~}
% for compatibility with earlier versions
\def\PYZat{@}
\def\PYZlb{[}
\def\PYZrb{]}
\makeatother


    % Exact colors from NB
    \definecolor{incolor}{rgb}{0.0, 0.0, 0.5}
    \definecolor{outcolor}{rgb}{0.545, 0.0, 0.0}



    
    % Prevent overflowing lines due to hard-to-break entities
    \sloppy 
    % Setup hyperref package
    \hypersetup{
      breaklinks=true,  % so long urls are correctly broken across lines
      colorlinks=true,
      urlcolor=urlcolor,
      linkcolor=linkcolor,
      citecolor=citecolor,
      }
    % Slightly bigger margins than the latex defaults
    
    \geometry{verbose,tmargin=1in,bmargin=1in,lmargin=1in,rmargin=1in}
    
    

    \begin{document}
    
    
    \maketitle
    
    

    
    \textbf{Name: Christ-Brian Amedjonekou}\\
\textbf{Date: 4/28/2019}\\
\textbf{TCET 3102-E316 (Analog and Digital Communications) Lab 4}\\
\textbf{Spring 2019, Section: E316, Code: 37251}\\
\textbf{Instructor: Song Tang}

    \hypertarget{objective}{%
\subsubsection{Objective}\label{objective}}

\begin{itemize}
\tightlist
\item
  Design a bandpass (Butterworth) filter and utilize that filter to
  select/pass certain frequencies and reject other frequencies.
\end{itemize}

\hypertarget{equipment}{%
\subsubsection{Equipment}\label{equipment}}

\begin{itemize}
\tightlist
\item
  Computer Software
\end{itemize}

\hypertarget{theory}{%
\subsubsection{Theory}\label{theory}}

\begin{itemize}
\item
  Filters are used to remove unwanted parts of the input signal. In most
  cases, this is noise present outside the frequency band of the desired
  signal.
\item
  In many applications we need to filter out a particular band of
  frequencies. This is why we use a passband filter: to isolate a
  particular band of frequencies. With passband filters we have a upper
  and lower limit for the frequencies we allow to pass. Anything outside
  of these limits will be attenuated (filtered out/removed).
\item
  Pass-Band filters are usually created by combining High and Low Pass
  filters together.
\item
  Just like in other sciences wehave ideal and practical Pass-Band
  Filters.
\end{itemize}

\hypertarget{ideal-case}{%
\paragraph{Ideal Case}\label{ideal-case}}

\begin{itemize}
\tightlist
\item
  An ideal Pass-Band Filter will have no ripples (completely flat)
\item
  It would also attentuate all frequencies outside of its upper and
  lower limit (pass-band/band-pass)
\item
  It would transition from passband to stopband instantaneously
\item
  In the real world there are no ideal cases
\end{itemize}

\hypertarget{practical-real-world-case}{%
\paragraph{Practical (Real World)
Case}\label{practical-real-world-case}}

\begin{itemize}
\tightlist
\item
  A practical Pass-Band Filter will have some ripples (not completely
  flat)
\item
  It would fail attentuate all frequencies outside of its upper and
  lower limit (pass-band/band-pass)

  \begin{itemize}
  \tightlist
  \item
    This situation is call \textbf{\emph{`Filter Roll-Off'}}
  \end{itemize}
\item
  It would not transition from passband to stopband instantaneously
\item
  Observed/used In the real world all the time.
\end{itemize}

\hypertarget{how-to-handle-these-imperfections}{%
\subparagraph{How to Handle these
imperfections}\label{how-to-handle-these-imperfections}}

\begin{itemize}
\tightlist
\item
  Usually in design, engineers try to make \textbf{\emph{`Filter
  Roll-Off'}} as narrow as possible
\item
  This is done to allow close to ideal characteristics for filter
  functionality
\end{itemize}

    \hypertarget{imported-packages}{%
\subsubsection{Imported Packages}\label{imported-packages}}

    \begin{Verbatim}[commandchars=\\\{\}]
{\color{incolor}In [{\color{incolor}1}]:} \PY{c+c1}{\PYZsh{} These are the packages I\PYZsq{}ll need to solve this problem}
        \PY{k+kn}{import} \PY{n+nn}{math} \PY{k}{as} \PY{n+nn}{ma}
        \PY{k+kn}{import} \PY{n+nn}{numpy} \PY{k}{as} \PY{n+nn}{np}
        \PY{k+kn}{from} \PY{n+nn}{matplotlib} \PY{k}{import} \PY{n}{pyplot} \PY{k}{as} \PY{n}{plt}
        \PY{k+kn}{from} \PY{n+nn}{scipy}\PY{n+nn}{.}\PY{n+nn}{fftpack} \PY{k}{import} \PY{n}{fft}\PY{p}{,} \PY{n}{fftfreq}
        \PY{k+kn}{from} \PY{n+nn}{scipy}\PY{n+nn}{.}\PY{n+nn}{signal} \PY{k}{import} \PY{n}{butter}\PY{p}{,} \PY{n}{buttord}\PY{p}{,} \PY{n}{freqz}\PY{p}{,} \PY{n}{lfilter}
\end{Verbatim}


    \hypertarget{run-1-generate-a-noisy-signal}{%
\subsubsection{RUN 1: Generate a Noisy
Signal}\label{run-1-generate-a-noisy-signal}}

    \begin{itemize}
\tightlist
\item
  \textbf{Step 1: Creating the Noisy Signal}
\end{itemize}

    \begin{Verbatim}[commandchars=\\\{\}]
{\color{incolor}In [{\color{incolor}2}]:} \PY{c+c1}{\PYZsh{} The code below this comment is \PYZsq{}Step 1\PYZsq{} of Run 1}
        \PY{c+c1}{\PYZsh{} Sampling Frequency}
        \PY{n}{Fs} \PY{o}{=} \PY{l+m+mi}{1000}
        
        \PY{c+c1}{\PYZsh{} Sampling Time (T), Length of signal (L)}
        \PY{n}{T}\PY{p}{,} \PY{n}{L} \PY{o}{=} \PY{l+m+mi}{1}\PY{o}{/}\PY{n}{Fs}\PY{p}{,} \PY{l+m+mi}{1000}
        
        \PY{c+c1}{\PYZsh{} Time Vector (t)}
        \PY{n}{t} \PY{o}{=} \PY{n}{np}\PY{o}{.}\PY{n}{linspace}\PY{p}{(}\PY{l+m+mi}{0}\PY{p}{,} \PY{n}{L}\PY{o}{\PYZhy{}}\PY{l+m+mi}{1}\PY{p}{,} \PY{l+m+mi}{1000}\PY{p}{)}\PY{o}{*}\PY{n}{T}
        
        \PY{c+c1}{\PYZsh{} We set \PYZsq{}x\PYZsq{} to be the sum of 50 Hz and 100 Hz sinusiods}
        \PY{n}{c}\PY{p}{,} \PY{n}{f} \PY{o}{=} \PY{p}{[}\PY{l+m+mf}{0.6}\PY{p}{,} \PY{l+m+mf}{0.9}\PY{p}{,} \PY{l+m+mf}{1.2}\PY{p}{]}\PY{p}{,} \PY{p}{[}\PY{l+m+mi}{50}\PY{p}{,} \PY{l+m+mi}{100}\PY{p}{]}
        \PY{n}{x} \PY{o}{=} \PY{n}{c}\PY{p}{[}\PY{l+m+mi}{0}\PY{p}{]}\PY{o}{*}\PY{n}{np}\PY{o}{.}\PY{n}{sin}\PY{p}{(}\PY{l+m+mi}{2}\PY{o}{*}\PY{n}{np}\PY{o}{.}\PY{n}{pi}\PY{o}{*}\PY{n}{f}\PY{p}{[}\PY{l+m+mi}{0}\PY{p}{]}\PY{o}{*}\PY{n}{t}\PY{p}{)} \PY{o}{+} \PY{n}{c}\PY{p}{[}\PY{l+m+mi}{1}\PY{p}{]}\PY{o}{*}\PY{n}{np}\PY{o}{.}\PY{n}{sin}\PY{p}{(}\PY{l+m+mi}{2}\PY{o}{*}\PY{n}{np}\PY{o}{.}\PY{n}{pi}\PY{o}{*}\PY{n}{f}\PY{p}{[}\PY{l+m+mi}{1}\PY{p}{]}\PY{o}{*}\PY{n}{t}\PY{p}{)}
        
        \PY{c+c1}{\PYZsh{} Noisy signal Creation}
        \PY{n}{noise} \PY{o}{=} \PY{n}{c}\PY{p}{[}\PY{l+m+mi}{2}\PY{p}{]}\PY{o}{*}\PY{n}{np}\PY{o}{.}\PY{n}{random}\PY{o}{.}\PY{n}{randn}\PY{p}{(}\PY{n}{t}\PY{o}{.}\PY{n}{size}\PY{p}{)}
        
        \PY{c+c1}{\PYZsh{} Signal w/ Noise}
        \PY{n}{swN} \PY{o}{=} \PY{n}{x} \PY{o}{+} \PY{n}{noise}
        
        \PY{c+c1}{\PYZsh{} Plots of Signal w/ Noise}
        \PY{n}{plt}\PY{o}{.}\PY{n}{figure}\PY{p}{(}\PY{n}{figsize}\PY{o}{=} \PY{p}{(}\PY{l+m+mi}{10}\PY{p}{,}\PY{l+m+mi}{5}\PY{p}{)}\PY{p}{)}
        \PY{n}{plt}\PY{o}{.}\PY{n}{plot}\PY{p}{(}\PY{n}{Fs}\PY{o}{*}\PY{n}{t}\PY{p}{[}\PY{p}{:}\PY{l+m+mi}{50}\PY{p}{]}\PY{p}{,} \PY{n}{swN}\PY{p}{[}\PY{p}{:}\PY{l+m+mi}{50}\PY{p}{]}\PY{p}{)}
        \PY{n}{plt}\PY{o}{.}\PY{n}{title}\PY{p}{(}\PY{l+s+s1}{\PYZsq{}}\PY{l+s+s1}{Signal corrupted w/ Random Noise}\PY{l+s+s1}{\PYZsq{}}\PY{p}{)}
        \PY{n}{plt}\PY{o}{.}\PY{n}{xlabel}\PY{p}{(}\PY{l+s+s1}{\PYZsq{}}\PY{l+s+s1}{Time (Milliseconds)}\PY{l+s+s1}{\PYZsq{}}\PY{p}{)}
\end{Verbatim}


\begin{Verbatim}[commandchars=\\\{\}]
{\color{outcolor}Out[{\color{outcolor}2}]:} Text(0.5, 0, 'Time (Milliseconds)')
\end{Verbatim}
            
    \begin{center}
    \adjustimage{max size={0.9\linewidth}{0.9\paperheight}}{output_6_1.png}
    \end{center}
    { \hspace*{\fill} \\}
    
    \begin{itemize}
\tightlist
\item
  \textbf{Step 2: Observing the Noisy Signal}
\end{itemize}

\begin{Shaded}
\begin{Highlighting}[]
\CommentTok{# Also the Next Power of 2 from Length 'y' function definition}
\KeywordTok{def}\NormalTok{ nextpow2(i):}
    \CommentTok{""" This is internal function used by fft(), because the FFT routine}
\CommentTok{   requires that the data size be a power of 2."""}
\NormalTok{    n }\OperatorTok{=} \DecValTok{1}
    \ControlFlowTok{while}\NormalTok{ n }\OperatorTok{<}\NormalTok{ i: }
\NormalTok{        n }\OperatorTok{*=} \DecValTok{2}
    \ControlFlowTok{return}\NormalTok{ n}
\end{Highlighting}
\end{Shaded}

    \begin{Verbatim}[commandchars=\\\{\}]
{\color{incolor}In [{\color{incolor}3}]:} \PY{c+c1}{\PYZsh{} The code below this comment is \PYZsq{}Step 2\PYZsq{} of Run 1}
        \PY{c+c1}{\PYZsh{} Next Power of 2 from Length \PYZsq{}y\PYZsq{} function definition}
        \PY{k}{def} \PY{n+nf}{nextpow2}\PY{p}{(}\PY{n}{i}\PY{p}{)}\PY{p}{:}
            \PY{l+s+sd}{\PYZdq{}\PYZdq{}\PYZdq{} This is internal function used by fft(), because the FFT routine}
        \PY{l+s+sd}{   requires that the data size be a power of 2.\PYZdq{}\PYZdq{}\PYZdq{}}
            \PY{k}{return} \PY{l+m+mi}{1} \PY{k}{if} \PY{n}{i} \PY{o}{==} \PY{l+m+mi}{0} \PY{k}{else} \PY{l+m+mi}{2}\PY{o}{*}\PY{o}{*}\PY{p}{(}\PY{n}{i} \PY{o}{\PYZhy{}} \PY{l+m+mi}{1}\PY{p}{)}\PY{o}{.}\PY{n}{bit\PYZus{}length}\PY{p}{(}\PY{p}{)}
        
        \PY{c+c1}{\PYZsh{} Next Power of 2 from Length \PYZsq{}1000\PYZsq{}}
        \PY{n}{NFFT} \PY{o}{=} \PY{l+m+mi}{2}\PY{o}{\PYZca{}}\PY{n}{nextpow2}\PY{p}{(}\PY{n}{L}\PY{p}{)}
        
        \PY{c+c1}{\PYZsh{} Fast Fourier Transform of the Noisy Signal}
        \PY{n}{s} \PY{o}{=} \PY{n}{fft}\PY{p}{(}\PY{n}{swN}\PY{p}{,} \PY{n}{NFFT}\PY{p}{)}\PY{o}{/}\PY{n}{L}
        \PY{n}{f\PYZus{}} \PY{o}{=} \PY{n}{Fs}\PY{o}{/}\PY{l+m+mi}{2} \PY{o}{*} \PY{n}{np}\PY{o}{.}\PY{n}{linspace}\PY{p}{(}\PY{l+m+mi}{0}\PY{p}{,} \PY{l+m+mi}{1}\PY{p}{,} \PY{n+nb}{int}\PY{p}{(}\PY{n}{NFFT}\PY{o}{/}\PY{l+m+mi}{2}\PY{o}{+}\PY{l+m+mi}{1}\PY{p}{)}\PY{p}{)}
        
        \PY{c+c1}{\PYZsh{} Plots of Signal w/ Noise}
        \PY{n}{plt}\PY{o}{.}\PY{n}{figure}\PY{p}{(}\PY{n}{figsize}\PY{o}{=} \PY{p}{(}\PY{l+m+mi}{10}\PY{p}{,}\PY{l+m+mi}{5}\PY{p}{)}\PY{p}{)}
        \PY{n}{plt}\PY{o}{.}\PY{n}{plot}\PY{p}{(}\PY{n}{f\PYZus{}}\PY{p}{,} \PY{l+m+mi}{2}\PY{o}{*}\PY{n+nb}{abs}\PY{p}{(}\PY{n}{s}\PY{p}{[}\PY{p}{:}\PY{n+nb}{int}\PY{p}{(}\PY{n}{NFFT}\PY{o}{/}\PY{l+m+mi}{2}\PY{o}{+}\PY{l+m+mi}{1}\PY{p}{)}\PY{p}{]}\PY{p}{)}\PY{p}{)}
        \PY{n}{plt}\PY{o}{.}\PY{n}{title}\PY{p}{(}\PY{l+s+s1}{\PYZsq{}}\PY{l+s+s1}{Amplitude Spectrum of the Noisy Signal}\PY{l+s+s1}{\PYZsq{}}\PY{p}{)}
        \PY{n}{plt}\PY{o}{.}\PY{n}{xlabel}\PY{p}{(}\PY{l+s+s1}{\PYZsq{}}\PY{l+s+s1}{Frequency (Hertz)}\PY{l+s+s1}{\PYZsq{}}\PY{p}{)}
        \PY{n}{plt}\PY{o}{.}\PY{n}{ylabel}\PY{p}{(}\PY{l+s+s1}{\PYZsq{}}\PY{l+s+s1}{|S(F)|}\PY{l+s+s1}{\PYZsq{}}\PY{p}{)}
        \PY{n}{plt}\PY{o}{.}\PY{n}{grid}\PY{p}{(}\PY{n}{which}\PY{o}{=} \PY{l+s+s1}{\PYZsq{}}\PY{l+s+s1}{both}\PY{l+s+s1}{\PYZsq{}}\PY{p}{)}
\end{Verbatim}


    \begin{center}
    \adjustimage{max size={0.9\linewidth}{0.9\paperheight}}{output_8_0.png}
    \end{center}
    { \hspace*{\fill} \\}
    
    \hypertarget{run-2-design-a-pass-band-butterworth-filter}{%
\subsubsection{RUN 2: Design a Pass-Band (Butterworth)
Filter}\label{run-2-design-a-pass-band-butterworth-filter}}

\begin{itemize}
\tightlist
\item
  \textbf{Step 1: Designing the Bandpass Filter}
\end{itemize}

    \begin{Verbatim}[commandchars=\\\{\}]
{\color{incolor}In [{\color{incolor}12}]:} \PY{c+c1}{\PYZsh{} The code below this comment is \PYZsq{}Step 1\PYZsq{} of Run 2}
         \PY{c+c1}{\PYZsh{} Nyquist Frequency}
         \PY{n}{Fn} \PY{o}{=} \PY{n}{Fs}\PY{o}{/}\PY{l+m+mi}{2}
         
         \PY{c+c1}{\PYZsh{} Bassband (Wp) and Stopband (Ws) frequencies normalized to Nyquist Frequencies}
         \PY{c+c1}{\PYZsh{} Passband (Rp) and Stopband (Rs) ripples}
         \PY{n}{Wp}\PY{p}{,} \PY{n}{Ws}\PY{p}{,} \PY{n}{Rp}\PY{p}{,} \PY{n}{Rs} \PY{o}{=} \PY{n}{np}\PY{o}{.}\PY{n}{array}\PY{p}{(}\PY{p}{[}\PY{l+m+mi}{40}\PY{p}{,} \PY{l+m+mi}{60}\PY{p}{]}\PY{p}{)}\PY{o}{/}\PY{n}{Fn}\PY{p}{,} \PY{n}{np}\PY{o}{.}\PY{n}{array}\PY{p}{(}\PY{p}{[}\PY{l+m+mi}{30}\PY{p}{,} \PY{l+m+mi}{70}\PY{p}{]}\PY{p}{)}\PY{o}{/}\PY{n}{Fn}\PY{p}{,} \PY{l+m+mi}{2}\PY{p}{,} \PY{l+m+mi}{35}
         
         \PY{c+c1}{\PYZsh{} Finds the filter order for a given stopband and cuttoff frequencies}
         \PY{n}{N}\PY{p}{,} \PY{n}{Wn} \PY{o}{=} \PY{n}{buttord}\PY{p}{(}\PY{n}{Wp}\PY{p}{,} \PY{n}{Ws}\PY{p}{,} \PY{n}{Rp}\PY{p}{,} \PY{n}{Rs}\PY{p}{)}
         \PY{n}{b}\PY{p}{,} \PY{n}{a} \PY{o}{=} \PY{n}{butter}\PY{p}{(}\PY{n}{N}\PY{p}{,} \PY{n}{Wn}\PY{p}{,} \PY{n}{btype}\PY{o}{=} \PY{l+s+s1}{\PYZsq{}}\PY{l+s+s1}{bandpass}\PY{l+s+s1}{\PYZsq{}}\PY{p}{)}
         \PY{n}{omega\PYZus{}}\PY{p}{,} \PY{n}{H} \PY{o}{=} \PY{n}{freqz}\PY{p}{(}\PY{n}{b}\PY{p}{,} \PY{n}{a}\PY{p}{,} \PY{l+m+mi}{1024}\PY{p}{,} \PY{n}{fs}\PY{o}{=} \PY{n}{Fs}\PY{p}{)}
         \PY{n}{xval} \PY{o}{=} \PY{p}{(}\PY{n}{omega\PYZus{}} \PY{o}{*} \PY{n}{Fs}\PY{p}{)}\PY{o}{/}\PY{p}{(}\PY{l+m+mi}{2} \PY{o}{*} \PY{n}{np}\PY{o}{.}\PY{n}{pi}\PY{p}{)}
         \PY{n}{yval} \PY{o}{=} \PY{n+nb}{abs}\PY{p}{(}\PY{n}{H}\PY{p}{)}
         
         \PY{c+c1}{\PYZsh{} Plot the Band Pass Filter}
         \PY{n}{plt}\PY{o}{.}\PY{n}{figure}\PY{p}{(}\PY{n}{figsize}\PY{o}{=} \PY{p}{(}\PY{l+m+mi}{10}\PY{p}{,} \PY{l+m+mi}{5}\PY{p}{)}\PY{p}{)}
         \PY{n}{plt}\PY{o}{.}\PY{n}{plot}\PY{p}{(}\PY{n}{xval}\PY{p}{,} \PY{n}{yval}\PY{p}{)}
         \PY{n}{plt}\PY{o}{.}\PY{n}{xlabel}\PY{p}{(}\PY{l+s+s1}{\PYZsq{}}\PY{l+s+s1}{Frequency in Hz}\PY{l+s+s1}{\PYZsq{}}\PY{p}{)}
         \PY{n}{plt}\PY{o}{.}\PY{n}{ylabel}\PY{p}{(}\PY{l+s+s1}{\PYZsq{}}\PY{l+s+s1}{Amplitude}\PY{l+s+s1}{\PYZsq{}}\PY{p}{)}
         \PY{n}{plt}\PY{o}{.}\PY{n}{title}\PY{p}{(}\PY{l+s+s1}{\PYZsq{}}\PY{l+s+s1}{Band Pass Filter}\PY{l+s+s1}{\PYZsq{}}\PY{p}{)}
         \PY{n}{plt}\PY{o}{.}\PY{n}{xlim}\PY{p}{(}\PY{l+m+mi}{0}\PY{p}{,} \PY{l+m+mi}{15000}\PY{p}{)}
         \PY{n}{plt}\PY{o}{.}\PY{n}{grid}\PY{p}{(}\PY{p}{)}
\end{Verbatim}


    \begin{center}
    \adjustimage{max size={0.9\linewidth}{0.9\paperheight}}{output_10_0.png}
    \end{center}
    { \hspace*{\fill} \\}
    
    \begin{Verbatim}[commandchars=\\\{\}]
{\color{incolor}In [{\color{incolor}5}]:} \PY{p}{(}\PY{n}{xval}\PY{p}{,} \PY{n}{yval}\PY{o}{.}\PY{n}{size}\PY{p}{)}
\end{Verbatim}


\begin{Verbatim}[commandchars=\\\{\}]
{\color{outcolor}Out[{\color{outcolor}5}]:} (array([0.00000000e+00, 7.77123746e+01, 1.55424749e+02, {\ldots},
                7.93443344e+04, 7.94220468e+04, 7.94997592e+04]), 1024)
\end{Verbatim}
            
    \begin{itemize}
\tightlist
\item
  \textbf{Step 2: Filtering the 50 Hz Signal from the Noisy Signal in
  RUN 1}
\end{itemize}

    \begin{Verbatim}[commandchars=\\\{\}]
{\color{incolor}In [{\color{incolor}6}]:} \PY{c+c1}{\PYZsh{} The code below this comment is \PYZsq{}Step 2\PYZsq{} of Run 2}
        \PY{c+c1}{\PYZsh{} Filtered Signal in the time domain}
        \PY{n}{sf} \PY{o}{=} \PY{n}{lfilter}\PY{p}{(}\PY{n}{b}\PY{p}{,}\PY{n}{a}\PY{p}{,}\PY{n}{swN}\PY{p}{)}
        
        \PY{c+c1}{\PYZsh{} Plots of the filtered signal}
        \PY{n}{plt}\PY{o}{.}\PY{n}{figure}\PY{p}{(}\PY{n}{figsize}\PY{o}{=} \PY{p}{(}\PY{l+m+mi}{10}\PY{p}{,}\PY{l+m+mi}{5}\PY{p}{)}\PY{p}{)}
        \PY{n}{plt}\PY{o}{.}\PY{n}{plot}\PY{p}{(}\PY{n}{t}\PY{p}{,} \PY{n}{sf}\PY{p}{[}\PY{p}{:}\PY{l+m+mi}{1000}\PY{p}{]}\PY{p}{)}
        \PY{n}{plt}\PY{o}{.}\PY{n}{title}\PY{p}{(}\PY{l+s+s1}{\PYZsq{}}\PY{l+s+s1}{Filtered Signal in the time domain}\PY{l+s+s1}{\PYZsq{}}\PY{p}{)}
        \PY{n}{plt}\PY{o}{.}\PY{n}{xlabel}\PY{p}{(}\PY{l+s+s1}{\PYZsq{}}\PY{l+s+s1}{Time (Milliseconds)}\PY{l+s+s1}{\PYZsq{}}\PY{p}{)}
        \PY{n}{plt}\PY{o}{.}\PY{n}{ylabel}\PY{p}{(}\PY{l+s+s1}{\PYZsq{}}\PY{l+s+s1}{Filtered Signal}\PY{l+s+s1}{\PYZsq{}}\PY{p}{)}
\end{Verbatim}


\begin{Verbatim}[commandchars=\\\{\}]
{\color{outcolor}Out[{\color{outcolor}6}]:} Text(0, 0.5, 'Filtered Signal')
\end{Verbatim}
            
    \begin{center}
    \adjustimage{max size={0.9\linewidth}{0.9\paperheight}}{output_13_1.png}
    \end{center}
    { \hspace*{\fill} \\}
    
    \begin{Verbatim}[commandchars=\\\{\}]
{\color{incolor}In [{\color{incolor}7}]:} \PY{c+c1}{\PYZsh{} The code below this comment is \PYZsq{}Step 2\PYZsq{} of Run 2}
        \PY{c+c1}{\PYZsh{} Noisy Signal \PYZam{} Filtered Signal in the Frequency domain}
        \PY{n}{S} \PY{o}{=} \PY{n}{fft}\PY{p}{(}\PY{n}{swN}\PY{p}{,} \PY{l+m+mi}{1024}\PY{p}{)}
        \PY{n}{SF} \PY{o}{=} \PY{n}{fft}\PY{p}{(}\PY{n}{sf}\PY{p}{,} \PY{l+m+mi}{1024}\PY{p}{)}
        \PY{n}{\PYZus{}f} \PY{o}{=} \PY{n}{np}\PY{o}{.}\PY{n}{linspace}\PY{p}{(}\PY{l+m+mi}{0}\PY{p}{,} \PY{l+m+mi}{511}\PY{p}{,} \PY{l+m+mi}{512}\PY{p}{)}\PY{o}{/}\PY{l+m+mi}{512}\PY{o}{*}\PY{n}{Fn}
        
        \PY{c+c1}{\PYZsh{} Plots of Noisy Signal in Frequency Domain}
        \PY{n}{plt}\PY{o}{.}\PY{n}{figure}\PY{p}{(}\PY{n}{figsize}\PY{o}{=} \PY{p}{(}\PY{l+m+mi}{10}\PY{p}{,}\PY{l+m+mi}{10}\PY{p}{)}\PY{p}{)}
        \PY{n}{plt}\PY{o}{.}\PY{n}{subplot}\PY{p}{(}\PY{l+m+mi}{2}\PY{p}{,} \PY{l+m+mi}{1}\PY{p}{,} \PY{l+m+mi}{1}\PY{p}{)}
        \PY{n}{plt}\PY{o}{.}\PY{n}{plot}\PY{p}{(}\PY{n}{\PYZus{}f}\PY{p}{,} \PY{n+nb}{abs}\PY{p}{(}\PY{n}{S}\PY{p}{[}\PY{p}{:}\PY{l+m+mi}{512}\PY{p}{]}\PY{p}{)}\PY{p}{)}
        \PY{n}{plt}\PY{o}{.}\PY{n}{title}\PY{p}{(}\PY{l+s+s1}{\PYZsq{}}\PY{l+s+s1}{Noisy Signal in Frequency Domain}\PY{l+s+s1}{\PYZsq{}}\PY{p}{)}
        \PY{n}{plt}\PY{o}{.}\PY{n}{xlabel}\PY{p}{(}\PY{l+s+s1}{\PYZsq{}}\PY{l+s+s1}{Frequency (Hertz)}\PY{l+s+s1}{\PYZsq{}}\PY{p}{)}
        \PY{n}{plt}\PY{o}{.}\PY{n}{ylabel}\PY{p}{(}\PY{l+s+s1}{\PYZsq{}}\PY{l+s+s1}{|S(F)|}\PY{l+s+s1}{\PYZsq{}}\PY{p}{)}
        \PY{n}{plt}\PY{o}{.}\PY{n}{grid}\PY{p}{(}\PY{n}{which}\PY{o}{=} \PY{l+s+s1}{\PYZsq{}}\PY{l+s+s1}{both}\PY{l+s+s1}{\PYZsq{}}\PY{p}{)}
        
        \PY{c+c1}{\PYZsh{} Plots of Filtered Signal in the Frequency domain}
        \PY{n}{plt}\PY{o}{.}\PY{n}{subplot}\PY{p}{(}\PY{l+m+mi}{2}\PY{p}{,} \PY{l+m+mi}{1}\PY{p}{,} \PY{l+m+mi}{2}\PY{p}{)}
        \PY{n}{plt}\PY{o}{.}\PY{n}{plot}\PY{p}{(}\PY{n}{\PYZus{}f}\PY{p}{,} \PY{n+nb}{abs}\PY{p}{(}\PY{n}{SF}\PY{p}{[}\PY{p}{:}\PY{l+m+mi}{512}\PY{p}{]}\PY{p}{)}\PY{p}{)}
        \PY{n}{plt}\PY{o}{.}\PY{n}{title}\PY{p}{(}\PY{l+s+s1}{\PYZsq{}}\PY{l+s+s1}{Filtered Signal in the Frequency domain}\PY{l+s+s1}{\PYZsq{}}\PY{p}{)}
        \PY{n}{plt}\PY{o}{.}\PY{n}{xlabel}\PY{p}{(}\PY{l+s+s1}{\PYZsq{}}\PY{l+s+s1}{Frequency (Hertz)}\PY{l+s+s1}{\PYZsq{}}\PY{p}{)}
        \PY{n}{plt}\PY{o}{.}\PY{n}{ylabel}\PY{p}{(}\PY{l+s+s1}{\PYZsq{}}\PY{l+s+s1}{|SF(F)|}\PY{l+s+s1}{\PYZsq{}}\PY{p}{)}
        \PY{n}{plt}\PY{o}{.}\PY{n}{grid}\PY{p}{(}\PY{n}{which}\PY{o}{=} \PY{l+s+s1}{\PYZsq{}}\PY{l+s+s1}{both}\PY{l+s+s1}{\PYZsq{}}\PY{p}{)}
\end{Verbatim}


    \begin{center}
    \adjustimage{max size={0.9\linewidth}{0.9\paperheight}}{output_14_0.png}
    \end{center}
    { \hspace*{\fill} \\}
    
    \hypertarget{questions}{%
\subsubsection{Questions}\label{questions}}

\textbf{1. Change passband frequencies to (90 Hz - 110 Hz) and maximum
passband ripple to 3 dB with 40 dB attenuation.}

    \begin{Verbatim}[commandchars=\\\{\}]
{\color{incolor}In [{\color{incolor}13}]:} \PY{c+c1}{\PYZsh{} The code below this comment is \PYZsq{}Step 1\PYZsq{} of Run 2}
         \PY{c+c1}{\PYZsh{} Nyquist Frequency}
         \PY{n}{Fn} \PY{o}{=} \PY{n}{Fs}\PY{o}{/}\PY{l+m+mi}{2}
         
         \PY{c+c1}{\PYZsh{} Bassband (Wp) and Stopband (Ws) frequencies normalized to Nyquist Frequencies}
         \PY{c+c1}{\PYZsh{} Passband (Rp) and Stopband (Rs) ripples}
         \PY{n}{Wp}\PY{p}{,} \PY{n}{Ws}\PY{p}{,} \PY{n}{Rp}\PY{p}{,} \PY{n}{Rs} \PY{o}{=} \PY{n}{np}\PY{o}{.}\PY{n}{array}\PY{p}{(}\PY{p}{[}\PY{l+m+mi}{90}\PY{p}{,} \PY{l+m+mi}{110}\PY{p}{]}\PY{p}{)}\PY{o}{/}\PY{n}{Fn}\PY{p}{,} \PY{n}{np}\PY{o}{.}\PY{n}{array}\PY{p}{(}\PY{p}{[}\PY{l+m+mi}{80}\PY{p}{,} \PY{l+m+mi}{120}\PY{p}{]}\PY{p}{)}\PY{o}{/}\PY{n}{Fn}\PY{p}{,} \PY{l+m+mi}{3}\PY{p}{,} \PY{l+m+mi}{40}
         
         \PY{c+c1}{\PYZsh{} Finds the filter order for a given stopband and cuttoff frequencies}
         \PY{n}{N}\PY{p}{,} \PY{n}{Wn} \PY{o}{=} \PY{n}{buttord}\PY{p}{(}\PY{n}{Wp}\PY{p}{,} \PY{n}{Ws}\PY{p}{,} \PY{n}{Rp}\PY{p}{,} \PY{n}{Rs}\PY{p}{)}
         \PY{n}{b}\PY{p}{,} \PY{n}{a} \PY{o}{=} \PY{n}{butter}\PY{p}{(}\PY{n}{N}\PY{p}{,} \PY{n}{Wn}\PY{p}{,} \PY{n}{btype}\PY{o}{=} \PY{l+s+s1}{\PYZsq{}}\PY{l+s+s1}{bandpass}\PY{l+s+s1}{\PYZsq{}}\PY{p}{)}
         \PY{n}{omega\PYZus{}}\PY{p}{,} \PY{n}{H} \PY{o}{=} \PY{n}{freqz}\PY{p}{(}\PY{n}{b}\PY{p}{,} \PY{n}{a}\PY{p}{,} \PY{l+m+mi}{1024}\PY{p}{,} \PY{n}{fs}\PY{o}{=} \PY{n}{Fs}\PY{p}{)}
         \PY{n}{xval} \PY{o}{=} \PY{p}{(}\PY{n}{omega\PYZus{}} \PY{o}{*} \PY{n}{Fs}\PY{p}{)}\PY{o}{/}\PY{p}{(}\PY{l+m+mi}{2} \PY{o}{*} \PY{n}{np}\PY{o}{.}\PY{n}{pi}\PY{p}{)}
         \PY{n}{yval} \PY{o}{=} \PY{n+nb}{abs}\PY{p}{(}\PY{n}{H}\PY{p}{)}
         
         \PY{c+c1}{\PYZsh{} Plot the Band Pass Filter}
         \PY{n}{plt}\PY{o}{.}\PY{n}{figure}\PY{p}{(}\PY{n}{figsize}\PY{o}{=} \PY{p}{(}\PY{l+m+mi}{10}\PY{p}{,} \PY{l+m+mi}{5}\PY{p}{)}\PY{p}{)}
         \PY{n}{plt}\PY{o}{.}\PY{n}{plot}\PY{p}{(}\PY{n}{xval}\PY{p}{,} \PY{n}{yval}\PY{p}{)}
         \PY{n}{plt}\PY{o}{.}\PY{n}{xlabel}\PY{p}{(}\PY{l+s+s1}{\PYZsq{}}\PY{l+s+s1}{Frequency in Hz}\PY{l+s+s1}{\PYZsq{}}\PY{p}{)}
         \PY{n}{plt}\PY{o}{.}\PY{n}{ylabel}\PY{p}{(}\PY{l+s+s1}{\PYZsq{}}\PY{l+s+s1}{Amplitude}\PY{l+s+s1}{\PYZsq{}}\PY{p}{)}
         \PY{n}{plt}\PY{o}{.}\PY{n}{title}\PY{p}{(}\PY{l+s+s1}{\PYZsq{}}\PY{l+s+s1}{Band Pass Filter}\PY{l+s+s1}{\PYZsq{}}\PY{p}{)}
         \PY{n}{plt}\PY{o}{.}\PY{n}{xlim}\PY{p}{(}\PY{l+m+mi}{10000}\PY{p}{,} \PY{l+m+mi}{25000}\PY{p}{)}
         \PY{n}{plt}\PY{o}{.}\PY{n}{grid}\PY{p}{(}\PY{p}{)}
\end{Verbatim}


    \begin{center}
    \adjustimage{max size={0.9\linewidth}{0.9\paperheight}}{output_16_0.png}
    \end{center}
    { \hspace*{\fill} \\}
    
    \textbf{2. Change signal to 70 Hz snusoid of amplitude 1 and 100 Hz
sinusoid of amplitude 0.9}

    \begin{Verbatim}[commandchars=\\\{\}]
{\color{incolor}In [{\color{incolor}9}]:} \PY{c+c1}{\PYZsh{} The code below this comment is \PYZsq{}Step 1\PYZsq{} of Run 1}
        \PY{c+c1}{\PYZsh{} Sampling Frequency}
        \PY{n}{Fs} \PY{o}{=} \PY{l+m+mi}{1000}
        
        \PY{c+c1}{\PYZsh{} Sampling Time (T), Length of signal (L)}
        \PY{n}{T}\PY{p}{,} \PY{n}{L} \PY{o}{=} \PY{l+m+mi}{1}\PY{o}{/}\PY{n}{Fs}\PY{p}{,} \PY{l+m+mi}{1000}
        
        \PY{c+c1}{\PYZsh{} Time Vector (t)}
        \PY{n}{t} \PY{o}{=} \PY{n}{np}\PY{o}{.}\PY{n}{linspace}\PY{p}{(}\PY{l+m+mi}{0}\PY{p}{,} \PY{n}{L}\PY{o}{\PYZhy{}}\PY{l+m+mi}{1}\PY{p}{,} \PY{l+m+mi}{1000}\PY{p}{)}\PY{o}{*}\PY{n}{T}
        
        \PY{c+c1}{\PYZsh{} We set \PYZsq{}x\PYZsq{} to be the sum of 50 Hz and 100 Hz sinusiods}
        \PY{n}{c}\PY{p}{,} \PY{n}{f} \PY{o}{=} \PY{p}{[}\PY{l+m+mi}{1}\PY{p}{,} \PY{l+m+mf}{0.9}\PY{p}{,} \PY{l+m+mf}{1.2}\PY{p}{]}\PY{p}{,} \PY{p}{[}\PY{l+m+mi}{70}\PY{p}{,} \PY{l+m+mi}{100}\PY{p}{]}
        \PY{n}{x} \PY{o}{=} \PY{n}{c}\PY{p}{[}\PY{l+m+mi}{0}\PY{p}{]}\PY{o}{*}\PY{n}{np}\PY{o}{.}\PY{n}{sin}\PY{p}{(}\PY{l+m+mi}{2}\PY{o}{*}\PY{n}{np}\PY{o}{.}\PY{n}{pi}\PY{o}{*}\PY{n}{f}\PY{p}{[}\PY{l+m+mi}{0}\PY{p}{]}\PY{o}{*}\PY{n}{t}\PY{p}{)} \PY{o}{+} \PY{n}{c}\PY{p}{[}\PY{l+m+mi}{1}\PY{p}{]}\PY{o}{*}\PY{n}{np}\PY{o}{.}\PY{n}{sin}\PY{p}{(}\PY{l+m+mi}{2}\PY{o}{*}\PY{n}{np}\PY{o}{.}\PY{n}{pi}\PY{o}{*}\PY{n}{f}\PY{p}{[}\PY{l+m+mi}{1}\PY{p}{]}\PY{o}{*}\PY{n}{t}\PY{p}{)}
        
        \PY{c+c1}{\PYZsh{} Noisy signal Creation}
        \PY{n}{noise} \PY{o}{=} \PY{n}{c}\PY{p}{[}\PY{l+m+mi}{2}\PY{p}{]}\PY{o}{*}\PY{n}{np}\PY{o}{.}\PY{n}{random}\PY{o}{.}\PY{n}{randn}\PY{p}{(}\PY{n}{t}\PY{o}{.}\PY{n}{size}\PY{p}{)}
        
        \PY{c+c1}{\PYZsh{} Signal w/ Noise}
        \PY{n}{swN} \PY{o}{=} \PY{n}{x} \PY{o}{+} \PY{n}{noise}
        
        \PY{c+c1}{\PYZsh{} Plots of Signal w/ Noise}
        \PY{n}{plt}\PY{o}{.}\PY{n}{figure}\PY{p}{(}\PY{n}{figsize}\PY{o}{=} \PY{p}{(}\PY{l+m+mi}{10}\PY{p}{,}\PY{l+m+mi}{5}\PY{p}{)}\PY{p}{)}
        \PY{n}{plt}\PY{o}{.}\PY{n}{plot}\PY{p}{(}\PY{n}{Fs}\PY{o}{*}\PY{n}{t}\PY{p}{[}\PY{p}{:}\PY{l+m+mi}{50}\PY{p}{]}\PY{p}{,} \PY{n}{swN}\PY{p}{[}\PY{p}{:}\PY{l+m+mi}{50}\PY{p}{]}\PY{p}{)}
        \PY{n}{plt}\PY{o}{.}\PY{n}{title}\PY{p}{(}\PY{l+s+s1}{\PYZsq{}}\PY{l+s+s1}{Signal corrupted w/ Random Noise}\PY{l+s+s1}{\PYZsq{}}\PY{p}{)}
        \PY{n}{plt}\PY{o}{.}\PY{n}{xlabel}\PY{p}{(}\PY{l+s+s1}{\PYZsq{}}\PY{l+s+s1}{Time (Milliseconds)}\PY{l+s+s1}{\PYZsq{}}\PY{p}{)}
\end{Verbatim}


\begin{Verbatim}[commandchars=\\\{\}]
{\color{outcolor}Out[{\color{outcolor}9}]:} Text(0.5, 0, 'Time (Milliseconds)')
\end{Verbatim}
            
    \begin{center}
    \adjustimage{max size={0.9\linewidth}{0.9\paperheight}}{output_18_1.png}
    \end{center}
    { \hspace*{\fill} \\}
    
    \textbf{3. Filter the 100 Hz sinosoid signal from the noisy gnal and
show frequency domain figures.}

    \begin{Verbatim}[commandchars=\\\{\}]
{\color{incolor}In [{\color{incolor}10}]:} \PY{c+c1}{\PYZsh{} The code below this comment is \PYZsq{}Step 2\PYZsq{} of Run 2}
         \PY{c+c1}{\PYZsh{} Filtered Signal in the time domain}
         \PY{n}{sf} \PY{o}{=} \PY{n}{lfilter}\PY{p}{(}\PY{n}{b}\PY{p}{,}\PY{n}{a}\PY{p}{,}\PY{n}{swN}\PY{p}{)}
         
         \PY{c+c1}{\PYZsh{} Plots of the filtered signal}
         \PY{n}{plt}\PY{o}{.}\PY{n}{figure}\PY{p}{(}\PY{n}{figsize}\PY{o}{=} \PY{p}{(}\PY{l+m+mi}{10}\PY{p}{,}\PY{l+m+mi}{5}\PY{p}{)}\PY{p}{)}
         \PY{n}{plt}\PY{o}{.}\PY{n}{plot}\PY{p}{(}\PY{n}{t}\PY{p}{,} \PY{n}{sf}\PY{p}{[}\PY{p}{:}\PY{l+m+mi}{1000}\PY{p}{]}\PY{p}{)}
         \PY{n}{plt}\PY{o}{.}\PY{n}{title}\PY{p}{(}\PY{l+s+s1}{\PYZsq{}}\PY{l+s+s1}{Filtered Signal in the time domain}\PY{l+s+s1}{\PYZsq{}}\PY{p}{)}
         \PY{n}{plt}\PY{o}{.}\PY{n}{xlabel}\PY{p}{(}\PY{l+s+s1}{\PYZsq{}}\PY{l+s+s1}{Time (Milliseconds)}\PY{l+s+s1}{\PYZsq{}}\PY{p}{)}
         \PY{n}{plt}\PY{o}{.}\PY{n}{ylabel}\PY{p}{(}\PY{l+s+s1}{\PYZsq{}}\PY{l+s+s1}{Filtered Signal}\PY{l+s+s1}{\PYZsq{}}\PY{p}{)}
\end{Verbatim}


\begin{Verbatim}[commandchars=\\\{\}]
{\color{outcolor}Out[{\color{outcolor}10}]:} Text(0, 0.5, 'Filtered Signal')
\end{Verbatim}
            
    \begin{center}
    \adjustimage{max size={0.9\linewidth}{0.9\paperheight}}{output_20_1.png}
    \end{center}
    { \hspace*{\fill} \\}
    
    \begin{Verbatim}[commandchars=\\\{\}]
{\color{incolor}In [{\color{incolor}11}]:} \PY{c+c1}{\PYZsh{} The code below this comment is \PYZsq{}Step 2\PYZsq{} of Run 2}
         \PY{c+c1}{\PYZsh{} Noisy Signal \PYZam{} Filtered Signal in the Frequency domain}
         \PY{n}{S} \PY{o}{=} \PY{n}{fft}\PY{p}{(}\PY{n}{swN}\PY{p}{,} \PY{l+m+mi}{1024}\PY{p}{)}
         \PY{n}{SF} \PY{o}{=} \PY{n}{fft}\PY{p}{(}\PY{n}{sf}\PY{p}{,} \PY{l+m+mi}{1024}\PY{p}{)}
         \PY{n}{\PYZus{}f} \PY{o}{=} \PY{n}{np}\PY{o}{.}\PY{n}{linspace}\PY{p}{(}\PY{l+m+mi}{0}\PY{p}{,} \PY{l+m+mi}{511}\PY{p}{,} \PY{l+m+mi}{512}\PY{p}{)}\PY{o}{/}\PY{l+m+mi}{512}\PY{o}{*}\PY{n}{Fn}
         
         \PY{c+c1}{\PYZsh{} Plots of Noisy Signal in Frequency Domain}
         \PY{n}{plt}\PY{o}{.}\PY{n}{figure}\PY{p}{(}\PY{n}{figsize}\PY{o}{=} \PY{p}{(}\PY{l+m+mi}{10}\PY{p}{,}\PY{l+m+mi}{10}\PY{p}{)}\PY{p}{)}
         \PY{n}{plt}\PY{o}{.}\PY{n}{subplot}\PY{p}{(}\PY{l+m+mi}{2}\PY{p}{,} \PY{l+m+mi}{1}\PY{p}{,} \PY{l+m+mi}{1}\PY{p}{)}
         \PY{n}{plt}\PY{o}{.}\PY{n}{plot}\PY{p}{(}\PY{n}{\PYZus{}f}\PY{p}{,} \PY{n+nb}{abs}\PY{p}{(}\PY{n}{S}\PY{p}{[}\PY{p}{:}\PY{l+m+mi}{512}\PY{p}{]}\PY{p}{)}\PY{p}{)}
         \PY{n}{plt}\PY{o}{.}\PY{n}{title}\PY{p}{(}\PY{l+s+s1}{\PYZsq{}}\PY{l+s+s1}{Noisy Signal in Frequency Domain}\PY{l+s+s1}{\PYZsq{}}\PY{p}{)}
         \PY{n}{plt}\PY{o}{.}\PY{n}{xlabel}\PY{p}{(}\PY{l+s+s1}{\PYZsq{}}\PY{l+s+s1}{Frequency (Hertz)}\PY{l+s+s1}{\PYZsq{}}\PY{p}{)}
         \PY{n}{plt}\PY{o}{.}\PY{n}{ylabel}\PY{p}{(}\PY{l+s+s1}{\PYZsq{}}\PY{l+s+s1}{|S(F)|}\PY{l+s+s1}{\PYZsq{}}\PY{p}{)}
         \PY{n}{plt}\PY{o}{.}\PY{n}{grid}\PY{p}{(}\PY{n}{which}\PY{o}{=} \PY{l+s+s1}{\PYZsq{}}\PY{l+s+s1}{both}\PY{l+s+s1}{\PYZsq{}}\PY{p}{)}
         
         \PY{c+c1}{\PYZsh{} Plots of Filtered Signal in the Frequency domain}
         \PY{n}{plt}\PY{o}{.}\PY{n}{subplot}\PY{p}{(}\PY{l+m+mi}{2}\PY{p}{,} \PY{l+m+mi}{1}\PY{p}{,} \PY{l+m+mi}{2}\PY{p}{)}
         \PY{n}{plt}\PY{o}{.}\PY{n}{plot}\PY{p}{(}\PY{n}{\PYZus{}f}\PY{p}{,} \PY{n+nb}{abs}\PY{p}{(}\PY{n}{SF}\PY{p}{[}\PY{p}{:}\PY{l+m+mi}{512}\PY{p}{]}\PY{p}{)}\PY{p}{)}
         \PY{n}{plt}\PY{o}{.}\PY{n}{title}\PY{p}{(}\PY{l+s+s1}{\PYZsq{}}\PY{l+s+s1}{Filtered Signal in the Frequency domain}\PY{l+s+s1}{\PYZsq{}}\PY{p}{)}
         \PY{n}{plt}\PY{o}{.}\PY{n}{xlabel}\PY{p}{(}\PY{l+s+s1}{\PYZsq{}}\PY{l+s+s1}{Frequency (Hertz)}\PY{l+s+s1}{\PYZsq{}}\PY{p}{)}
         \PY{n}{plt}\PY{o}{.}\PY{n}{ylabel}\PY{p}{(}\PY{l+s+s1}{\PYZsq{}}\PY{l+s+s1}{|SF(F)|}\PY{l+s+s1}{\PYZsq{}}\PY{p}{)}
         \PY{n}{plt}\PY{o}{.}\PY{n}{grid}\PY{p}{(}\PY{n}{which}\PY{o}{=} \PY{l+s+s1}{\PYZsq{}}\PY{l+s+s1}{both}\PY{l+s+s1}{\PYZsq{}}\PY{p}{)}
\end{Verbatim}


    \begin{center}
    \adjustimage{max size={0.9\linewidth}{0.9\paperheight}}{output_21_0.png}
    \end{center}
    { \hspace*{\fill} \\}
    
    \hypertarget{analysis}{%
\subsubsection{Analysis}\label{analysis}}

\begin{itemize}
\tightlist
\item
  \textbf{In the first run, we were able to see the signal corrupted
  with random noise from 0 to 50 ms time frame; This is shown in the
  time domain. For the frequency domain we see the frequencies 50 and
  100 Hz of the original signal plus all the other smaller frequencies
  representing the noise. Following up w/ run 2, we were able create the
  Butterworth bandpass filter but this filter has some sort of ripple
  and is not completely flat. However, it is able to remove the random
  noise, shown later in Run 2. Running RUNs 1 \& 2 w/ different values
  we were able to see the signal corrupted with random noise from 0 to
  50 ms time frame; This is the same as before. We also see that the
  frequencies 90 and 100 Hz of the original signal plus all the other
  smaller frequencies representing the noise. This time our Butterworth
  bandpass filter is completely flat, and like before, only allows
  frequencies within the desired frequency range to pass, rejecting all
  others.}
\end{itemize}

\hypertarget{conclusion}{%
\subsubsection{Conclusion}\label{conclusion}}

\begin{itemize}
\tightlist
\item
  \textbf{I was successful in running the simluation of bandpass
  filters. I learned about functionality of bandpass filter; That it can
  accept range of accepted frequencies and reject all others by setting
  up the desired frequency frequency range.}
\end{itemize}


    % Add a bibliography block to the postdoc
    
    
    
    \end{document}
